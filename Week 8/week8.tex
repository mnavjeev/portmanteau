%! TEX root = /Users/manunavjeevan/Documents/UCLA/Third Year/Reading Group/wcep.tex

We now want to show a Donsker Theorem using bracketing numbers. The proof is more involved than for the Glivenko-Cantelli Theorem (Theorem~\ref{thm:vdv2.4.1}) using bracketing numbers, so we may not get too far into it here. 

To show this theorem we make minor use of the following related statements from VdV\&W.
\begin{lemma}[Problem 2.5.5 VdV\&W]
	\label{lemma:vdv2.5.5}
	If \(X\) is a positive random variable, \(\|X\|_2^2\leq \sup_{t>0}t\E X\mathds{1}\{X > t\} \leq  2 \|X\|_2^2\).
\end{lemma}

\begin{lemma}[Problem 2.5.6 VdV\&W]
	\label{lemma:vdv2.5.6}
	Any random variable \(X\) with a finite second moment satisfies \(E|X|\{X > t\}=o(t^{-1})\) as \(t\to\infty\).
\end{lemma}
\begin{lemma}[Equation 2.5.5 VdV\&W]
	\label{eq:vdv2.5.5}
	For a finite set \(\calF\) of cardinality \(\calF\geq 2\),
	\[
		\E\|\mathbb{G}_n\|_\calF \lesssim \max_f \frac{\|f\|_\infty}{\sqrt{n}}\log |\calF| + \max_f \|f\|_{P,2}\sqrt{\log|\calF|} 
	.\] 
\end{lemma}
\begin{theorem}[Donsker Theorem for Bracketing Numbers]
	\label{thm:donsker-bracketing}
	Let \(\calF\) be a class of measurable functions with an envelope \(F\) such that \(P^\star F^2 < \infty\) and
	\begin{equation}
		\label{eq:donsker-bracketing-condition}
		\int_{0}^{\infty} \sqrt{\log\calN_{[\hspace{0.1em}]}\left(\eps,\calF,L_2(P)\right)}\;d\eps <\infty
	\end{equation}
	then \(\calF\) is Donsker.
\end{theorem}
\begin{proof}
	The proof of this is roughly based on the steps in Theorem 2.5.6 in VanDerVaart and Wellner. I attempt to replicate the argument below:

	For each \(q\in\SN\) there is a partition \(\calF = \bigcup_{i=1}^{N_q} \calF_{qi}\) of \(\calF\) into \(N_q\) disjoint subsets such that 
	\begin{align*}
		\sum_{i=1}^{\infty} 2^{-q} \sqrt{\log N_q} &< \infty \\	
		\|\big(\sup_{f,g\in\calF_{qi}}|f-g|\big)^\star\|_{P,2} &< 2^{-q} \\
		\sup_{f,g\in\calF_{qi}} \|f-g\|_{P,2} &< 2^{-q}
	\end{align*}
	To see this, cover \(\calF\) with a minimal numbers of \(L_2(P)\) balls and \(L_2(P)\) brackets of size \(2^{-q}\), disjointify and take the intersection of the two partitions. By definition of the \(2^{-q}\) balls and \(2^{-q}\) brackets, the last two conditions hold. To see that the first condition holds note that the integral of the bracketing number being finite implies that the integral of the covering number is finite, and that \(\calN_{[\hspace{0.1em}]}\) and \(\calN\) are decreasing in \(\eps\). For any decreasing function \(f\) it is clear that
	\[
		\sum_{i=1}^n 2^{-q} f(q) \leq \int_0^\infty f(q)\;dq
	.\]
	which gives us the first condition. This sequence of partitions can, without loss of generality, be chosen as successive refinements.
	\footnote{To see this, construct a sequence of partitions \(\calF = \bigcup_{i=1}^{\bar N_q} \bar\calF_{qi}\) without this property. Next, take the partition at stage \(q\), \(\calF = \bigcup_{i=1}^{N_q}\calF_{qi}\) to consists of all intersections of the form \(\bigcap_{p=1}^q \bar\calF_{p,i_p}\) (basically take the intersection of all partitions up till \(q\)) so that \(|N_q| = \prod_{p=1}^q \bar N_p\). Using the inequality \(\big(\log\prod\bar N_p\big)^{1/2} \leq \sum \big(\log \bar N_p\big)^{1/2}\)  we get that:
	\begin{align*}
		\sum_{q=1}^\infty 2^{-q} \sqrt{\log N_q} &\leq \sum_{q=1}^\infty 2^{-q} \sum_{p=1}^q \log \sqrt{\bar N_p} \\
												 &= \sum_{q=1}^\infty 2^{-q} \sum_{p=1}^\infty 2^{-p}\sqrt{\log \bar N_p}\\
												 &< \infty
	\end{align*}
	The equality is a bit difficult to see but follows from the preceding line after rewriting the double summation as a triangular array. To simplify, consider an infinite sequence \(\{a_i\}_{i=1}^n\). Then \(\sum_{i=1}^\infty 2^{-i} \sum_{j=1}^i a_i  = a_1 + \frac{1}{2}(a_1 + a_2) + \frac{1}{2^{2}}(a_1 + a_2 + a_3) + \dots\). Rearrange to get the result. This gives us that the first condition still holds for the new nesting sequence of partitions. The second and third conditions trivially still hold as the new partitions are finer than the previous ones.  
	}

	For each \(q\) choose a fixed element \(f_{qi}\) from each partitioning set \(\calF_{qi}\) and define \(\pi_qf = f_{qi}\) if \(f \in \calF_{qi}\). Further define \(\Delta_q f = \sup_{f,g\in\calF_{q,i}} |f-g|^\star\) if \(f\in\calF_{qi}\).

	Note that \(\pi_q f\) and \(\Delta_q f\) take on one of \(N_q\) values as \(f\) ranges through \(\calF\). In view of Theorem~\ref{thm:vdv1.5.6} it suffices to show that the sequence \(\|\mathbb{G}_n(f-\pi_{q_0}f)\|_\calF\) converges in probability to zero as \(n\to\infty\) for an arbitrary \(q_0\) and then take \(q_0 \to \infty\).

	Define for each fixed \(n\) and \(q \geq q_0\) the following numbers and indicator-type functions:
	\begin{align*}
		a_q &= 2^{-q}/\sqrt{\log N_{q+1}} \\
		A_{q-1}f &= \mathds{1}\{\Delta_{q0}f \leq \sqrt n a_{q_0} \wedge \dots\wedge \Delta_{q-1}f \leq \sqrt{n} a_{q-1}\}\\
		B_q f &= \mathds{1}\{\Delta_{q_0}f \leq \sqrt{n}a_{q_0}\wedge\dots\wedge \Delta_{q-1}f \leq \sqrt{n}a_{q-1}\wedge \Delta_q f > \sqrt{n} a_q \}\\
		B_{q_0} &= \mathds{1}\{\delta_{q_0}f > \sqrt{n}a_{q_0}\} 
	\end{align*}
	Note that \(A_qf\) and \(B_qf\) are constant in \(f\) on each of the partitioning sets \(\calF_{qi}\) at level \(q\) because the partitions are nested.\footnote{Recall that \(\Delta_q f\) is the same for all elements in \(\calF_{q,i}\) and that \(\calF_{qi} \subset \calF_{(q-1)i}\)}

	Now, pointwise in \(x\), decompose
	\begin{align}
		\label{eq:decomposition}
		\tag{B-1}
		f - \pi_{q_0}f = (f-\pi_{q_0}f)B_{q_0}f + \sum_{q= q_0+1}^\infty (f-\pi_qf)B_qf + \sum_{q=q+1}^\infty (\pi_qf-\pi_{q-1}f)A_{q-1}f  
	\end{align}
	Here note that we are essentially decomposing the event space into \(B_{q_0}\) and \(B_{q_0}^c = A_{q_0}\). We can think of \(B_q\) as \(f\) being ``between'' \(A_{q-1}\) and \(A_{q}\). That is, if all the conditions for \(B_q\) hold except for the last one, then \(A_{q-1}\) is equal to \(1\) and \(B_q\) is equal to zero. Conversely if \(B_q\) is equal to one, then all the conditions for \(A_q\) hold except for the last one (\(\Delta_q g \leq \sqrt{n} a_{q}\)) so \(A_q\) is equal to zero and \(B_q\) is equal to one. Equivalently \(A_{q} + B_q = A_{q-1}\) or \(A_{q-1} - A_{q} = B_q\). Combine this with the fact that sets indicated \(A_q\) are nested and telescope to get the decomposition above.  

	Now we will apply the empirical process \(\mathbb{G}_n = \sqrt{n}\left(\P_n-P\right)\) to each of the terms in~\eqref{eq:decomposition} and take the suprema over \(f\in\calF\). We will show that each of the resulting 3 variables converge to zero in probability and then take \(q_0 \to \infty\).

	First, since \(|f-\pi_{q_0}f|B_{q_0}f \leq 2F\mathds{1}\{2F > \sqrt{n}a_{q_0}\}\) one has that\footnote{Recall the definition of the ``\(\calF\)'' norm, pull out the \(\sqrt{n}\), apply triangle inequality and note that \(\E^\star\P_n F \leq \E^\star F\) because \((S+ T)^\star \leq S^\star + T^\star\) for any functions \(S,T\). } 
	\[
		\E^\star\|\mathbb{G}_n(f-\pi_{q_0}f)B_{q_0}f\|_\calF \leq 4\sqrt{n}P^\star F\mathds{1}\{2F > \sqrt{n}a_{q_0}\} 
	.\] 
	The right hand size converges to zero as \(n\to\infty\) by Lemma~\ref{lemma:vdv2.5.6} because \(F\) has a finite second outer moment.

	By applying Lemma~\ref{lemma:vdv2.5.5} and noting that \(B_qf \leq \mathds{1}\{\Delta_q f > \sqrt{n}a_q\} \):
	\[
		\sqrt{n}a_q P\Delta_qfB_qf \leq  \sqrt{n}a_q P\Delta_qf \mathds{1}\{\Delta q f > \sqrt{n}a_q\}\leq  2\|\Delta_q f\|_2^2 \leq 2\,2^{-2q}
	.\] 
	Applying once that \(\Delta_{q-1}fB_qf\) is bounded by \(\sqrt{n}a_{q-1}\) for \(q > q_0\), multiplying and dividing by \(a_q\), and applying the inequality from above, we obtain the that inequality that we will need below:
	\[
		P\left(\Delta_qfB_qf\right)^2 \leq \sqrt{n}a_{q-1}P\Delta_qf\mathds{1}\{\Delta_qf>\sqrt{n}a_q\} \leq 2\frac{a_{q-1}}{a_q}2^{-2q} 
	.\]
	And now applying the triangle inequality, using that \(|ab| = \left||a|b\right|\) for \(a,b\in\SR\):
	\begin{align*}
		\E^\star\left\|\sum_{q=q_0+1}^\infty \mathbb{G}_n(f-\pi_q)B_qf\right\|_\calF 
		&\leq \sum_{q=q_0+1}^\infty
		\E^\star\left\|\mathbb{G}_n\Delta_qfB_qf\right\|_\calF\\
		\intertext{Now applying Lemma~\ref{eq:vdv2.5.5}. Note that 1) the implicit constant in the inequality can be taken to be universal by the nesting property of the subsets, 2) that \(\Delta_qf B_qf \leq \sqrt{n}a_{q-1}\), the bounds derived above, and 3) that the supremum is taken over \(N_q\) functions at each level \(q\):}
		&\lesssim \sum_{q=q_0+1}^\infty a_{q-1}\log N_q + 2^{-q}\sqrt{2\frac{a_{q-1}}{a_q}} \sqrt{\log{N_q}}
		\intertext{Note that \(a_q\) is decreasing in \(q\) (as \(q\) increases the top of the fraction is getting smaller and the bottom of the fraction is getting larger) so that the quotient can be replaced by it's square. Now use the definition of \(a_q\) to  bound this sum:}
		&\lesssim \sum_{q=q_0+1}^\infty 2^{-q}\sqrt{\log N_q}
	\end{align*}
	This bound is independent of \(n\) and converges to zero as \(q_0\to\infty\).

	To bound the third term note that there are at most \(N_q\) functions \((\pi_q-\pi_{q-1})f\)\footnote{each \(\pi_q\) has a specific \(\pi_{q-1}\) attached to it} and at most \(N_{q-1}\) functions \(A_{q-1}f\). Since the partitions are nested (\(\calF_{qi}\subset\calF_{q-1,i}\)), the function \(|\pi_qf-\pi_{q-1}f|A_{q-1}f\) is bounded by \(\Delta_{q-1}fA_{q-1}f\leq \sqrt{n}a_{q-1}\). Also by nesting, the \(L_2(P)\) norm of \(\pi_qf-\pi_{q-1}f\) is bounded by \(2^{-(q-1)}\). Apply the inequality in Lemma~\ref{eq:vdv2.5.5} to get
	\[
		\E^\star\left\|\sum_{q=q_0+1}^\infty \mathbb{G}_n(\pi_q-\pi_{q-1}f)A_{q-1}f\right\|_\calF \lesssim \sum_{q=q_0+1}^\infty a_{q-1}\log N_q + 2^{-q}\sqrt{\log N_q} 
	.\] 
	As before, this upper bound is independent of \(n\) and converges to zero as \(q_0\to\infty\).

	Putting this all together we get that as \(n,q_0\to\infty\) 
	\[
		\E^\star\sup_{f,g\in\calF}|f-g| \leq 2E^\star \|f-\pi_{q_0}f\|_\calF\longto 0
	.\]
	which allows us to apply Theorem~\ref{thm:vdv1.5.6} and establish asymptotic equicontinuity. Because the envelope has finite second outer moment we can apply CLT to the marginals to get convergence to a tight distribution and apply Theorem~\ref{thm:vdv1.5.4} to get weak convergence of the whole process \(\mathbb{G}_n\).
\end{proof}
\begin{remark*}[Comments on Donsker Theorems]
	Note that the integral in the bracketing Donsker Theorem above does not require taking the supremum over all finitely discrete probability measures \(Q\), instead we only deal with the underlying measure \(P\). The idea here is the bracketing gives us more control on the arguments. 

	Also, note that there are no measurability concerns for the bracketing Donsker theorem. This is because we did not have to use symmetrization, instead relying on Lemma~\ref{eq:vdv2.5.5}.
\end{remark*}
\begin{remark*}[Summary: Glivenko-Cantelli vs. Donsker]
	Conditions for:

	\underline{Glivenko-Cantelli:}	
	\begin{itemize}
		\item If \(\calN_{[\hspace{0.1em}]}(\eps,\calF,L_1(P))<\infty\) for every \(\eps\) then \(\calF\) is (P)-Glivenko-Cantelli.
		\item If \(\calF\) is P-measurable with envelope \(F\) satisfying \(P^\star F <\infty\) and for every \(M,\eps > 0\) we have that \(\log\calN(\eps,\calF_M,L_1(\P_n)) = o_{p^\star}(n)\) then \(\calF\) is (P)-Glivenko-Cantelli.
	\end{itemize}
	\underline{Donsker:}
	\begin{itemize}
		\item If \(\calF\) has envelope \(F\) with \(P^\star F^2 <\infty\) and 
			\[
				\int_{0}^{\infty} \sqrt{\log\calN_{[\hspace{0.1em}]}(\eps,\calF,L_2(P))}\;d\eps <\infty  
			.\] 
			then \(\calF\) is Donsker. 
		\item If \(\calF\) has envelop \(F\) with \(P^\star F^2 <\infty\), \(\calF_\infty^2\) and \(\calF_\delta\) are P-measurable for every \(\delta\) and the \eqref{eq:uniform-entropy-bound} is satisfied, that is 
		\[
			\int_{0}^{\infty} \sup_\calQ \sqrt{\log\calN(\eps\|F\|_{Q,2},\calF,L_2(Q))}\;d\eps < \infty
		.\]
		then \(\calF\) is Donsker.
	\end{itemize}
	The bracketing numbers give you pointwise control of functions between the brackets, this gives us cleaner conditions for Glivenko-Cantelli and Donsker that depend only on the true measure. The covering number proofs require symmetrization and then applying Fubini's theorem, so we need measurability assumptions. The tradeoff is that bracketing numbers are larger than covering numbers, so the conditions may be harder to satisfy.
\end{remark*}

We have now reduced the conditions that we need for uniform convergence to conditions in terms of bracketing/covering numbers. Next we will turn to verifying these conditions and applying the uniform convergence results. 
