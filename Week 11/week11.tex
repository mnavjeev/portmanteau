%! TEX root = /Users/manunavjeevan/Documents/UCLA/Third Year/Reading Group/wcep.tex
\newpage
\section{Delta Method and Applications to Statistics}
\label{sec:delta-applications}

We now look to apply the results of Section~\ref{sec:empirical-processes} to the Delta Method and other statistical problems. 

\subsection{Multiplier Central Limit Theorems}%
\label{subsec:mclt}

This section follows Chapter 2.9 in Van der Vaart and Wellner. 

With the notation \(Z_i = \delta_{X_i}-P\), the empirical central limit theorem can be written
\[
    \frac{1}{\sqrt{n}} \sum_{i=1}^n Z_i \wcov \mathbb{G}
.\] 
where \(\mathbb{G}\) is a tight stochastic process on \(\ell^\infty(\calF)\). In contrast, given i.i.d real-valued random variables \(\xi_1,\dots,\xi_n\) which are independent of \(Z_1,\dots,Z_n\), a multiplier central limit theorem would assert that 
\[
    \frac{1}{\sqrt{n}}\sum_{i=1}^n \xi_iZ_i\wcov \mathbb{G}
.\]
A deeper result is a \emph{conditional multiplier central limit theorem}, which asserts that for almost every sequence \(Z_1,Z_2,\dots\) 
\[
    \frac{1}{\sqrt{n}}\sum_{i=1}^n \xi_iZ_i\wcov \mathbb{G}
.\]

To establish these results we will make use of symmetrization results of the sort seen in Lemma~\ref{lemma:vdv2.3.1}. First define the ``2-1'' norm:\footnote{I have no idea what this is actually called.}
\begin{definition}[2-1 Norm]
	\label{def:21norm}
	For a random variable \(X:\Omega\to\SR\) define
	\[
		\|X\|_{2,1}=\int_{0}^{\infty} \sqrt{\Pr\left(|X|>x\right)}\,dx 
	.\] 
\end{definition}
\begin{lemma}[Generalized Symmetrization]
	\label{lemma:generalized-symmetrization}
	Let \(Z_1,\dots,Z_n\) be independent stochastic processes with mean zero and let \(\eps_1,\eps_2,\dots\) be independently generated Rademacher random variables.\footnote{Can basically just think of an independent stochastic process as independent data. Each data point represents a random functional on \(\calF\) (evaluate each function \(f\in\calF\) at \(Z_i\)).} Then:
	\[
	    \E^\star\Phi\bigg(\frac{1}{2}\bigg\|\sum_{i=1}^n \eps_iZ_i\bigg\|_\calF\bigg)
		\leq \E^\star\bigg(\big\|\sum_{i=1}^n Z_i\big\|_\calF\bigg)
		\leq \E^\star\bigg(2\bigg\|\sum_{i=1}^n \eps_i(Z_i-\ell_i)\bigg\|_\calF\bigg)
	.\]
	for every nondecreasing, convex \(\phi:\SR\to\SR\) and arbitrary functions \(\ell_i:\calF\to\SR\).
\end{lemma}
\begin{lemma}[Donkser Implication]
	\label{lemma:donsker-implication}
	Let \(Z_1,Z_2,\dots\) be i.i.d stochastic processes such that \(\sqrt{n}\sum_{i=1}^n Z_i\) converges weakly in \(\ell^\infty(\calF)\) to a tight Gaussian process. Then
	\[
		\lim_{x\to\infty}x^2\sup_n \P^\star\left(\frac{1}{\sqrt{n}}\bigg\|\sum_{i=1}^n Z_i\bigg\|_\calF > x\right) = 0
	.\] 
	In particular, the random variable \(\|Z_1\|^\star_\calF\) possesses a weak second moment.
\end{lemma}
\begin{lemma}[Alternative Weak Convergence Characterizations]
	\label{lemma:convergence-characterizations}
	Let \(Z_1,Z_2,\dots\) be i.i.d stochastic processes, linear in \(f\). Set \(\rho_Z(f,g) = \Var_Z\left(f-g\right)\) and \(\calF_\delta = \left\{f-g:\rho_Z(f,g)<\delta\right\}\). Then the following statements are equivalent and imply that the sequence \(\E^\star\|n^{-1/2}\sum_{i=1}^n Z_i\|_\calF^r\) converges to \(\E\|\mathbb{G}\|_\calF^r\) for every \(0<r<2\):
	\begin{enumerate}
		\item \(n^{-1/2}\sum_{i=1}^n Z_i\) converges weakly to a tight limit in \(\ell^\infty(\calF)\);
		\item \((\calF,\rho_Z)\) is totally bounded\footnote{See Definition~\ref{def:totally-bounded}.} and \(\|n^{-1/2}\sum_{i=1}^n Z_i\|_{\calF_{\delta_n}}\to_{p^\star}0\) for every \(\delta_n\downarrow 0\);
		\item \((\calF,\rho_Z)\) is totally bounded and \(\E^\star\|n^{-1/2}\sum_{i=1}^n Z_i\|_{\calF_{\delta_n}}\to0\).
	\end{enumerate}
\end{lemma}
\begin{lemma}[Multiplier Inequalities]
	\label{lemma:vdv2.9.1}
	Let \(Z_1,\dots,Z_n\) be i.i.d stochastic processes with \(\E^\star\|Z_i\|_\calF<\infty\) independent of the Rademacher random variables \(\eps_1,\dots,\eps_n\). Then, for every i.i.d sample \(\xi_1,\dots,\xi_n\) of mean-zero symmetric random variables independent of \(Z_1,\dots,Z_n\) and any \(1\leq n_0\leq n\):
	\begin{align*}
		\|\xi\|_1\E^\star\bigg\|\frac{1}{\sqrt{n}}\sum_{i=1}^n \eps_iZ_i\bigg\|_\calF 
		&\leq \E^\star\bigg\|\frac{1}{\sqrt{n}}\sum_{i=1}^n \xi_iZ_i\bigg\|_\calF \\
		&\leq (n_0-1)\E^\star\|Z_1\|_\calF\E\max_{1\leq i\leq n}\frac{|\xi_i|}{\sqrt{n}} \\ 
		&\;\;\;\;\;+ \|\xi\|_{2,1}\max_{n_0\leq k\leq n}\E^\star\bigg\|\frac{1}{\sqrt{k}}\sum_{i=n_0}^k\eps_iZ_i \bigg\|_\calF 
	\end{align*}
\end{lemma}

These lemmas are used to show the following theorem: 
\begin{theorem}[Unconditional Multiplier Central Limit Theorem]
	\label{thm:unconditional-multiplier-clt}
	Let \(\calF\) be a class of measurable functions. Let \(\xi_1,\dots,\xi_n\) be i.i.d symmetric random variables with mean zero, variance one, and \(\|\xi\|_{2,1} < \infty\), independent of \(X_1,\dots,X_n\). Then the sequence \(n^{-1/2}\sum_{i=1}^n \xi_1(\delta_{X_i}-P)\) converges to a tight limiting process in \(\ell^\infty(\calF)\) if and only if \(\calF\) is Donsker.
\end{theorem}
\begin{proof}
	Since we can replace any \(f\in\calF\) with \(f-Pf\) without changing the value of either the original or multiplier empirical process, it can be assumed without loss of generality that \(Pf=0\) for every \(f\). Marginal convergence of both sequences is equivalent to \(\calF\subset L_2(P)\). In light of Theorem~\ref{thm:vdv1.5.4} and Theorem~\ref{thm:vdv1.5.6}, it suffices to show that the asymptotic equicontinuity results for the empirical and multiplier processes are equivalent.

	If \(\calF\) is Donsker then \(\Pr^\star\left(F>x\right)=o(x^{-2})\) by Lemma~\ref{lemma:donsker-implication}. By the same lemma, convergence of the multiplier process to a tight limit implies that \(\Pr^\star\left(|\xi F|>x\right)=o(x^{-2})\). In particular, \(\P^\star F <\infty\) in both cases. 

	Since \(\|\xi\|_{2,1}<\infty\) implies the existence of a second moment, we have that \(\E^\star\max_{1\leq i\leq n}|\xi_i|/\sqrt{n}\to 0\) by Markov's Inequality. Applying Lemma~\ref{lemma:vdv2.9.1} gives
	\begin{align*}
	   \|\xi\|_1\limsup_{n\to\infty}\E^\star\bigg\|\frac{1}{\sqrt{n}}\sum_{i=1}^n \eps_iZ_i\bigg\|_{F_\delta} 	
	   &\leq \limsup_{n\to\infty}\E^\star\bigg\|\frac{1}{\sqrt{n}}\sum_{i=1}^n \xi_iZ_i\bigg\|_{\calF_\delta} \\
	   &\leq \|\xi\|_{2,1}\sup_{n_0\leq k}\E^\star\bigg\|\frac{1}{\sqrt{k}}\sum_{i=1}^k\eps_iZ_i\bigg\|_{\calF_\delta}
	\end{align*}
	for every \(n_0\) and \(\delta > 0\). By Lemma~\ref{lemma:generalized-symmetrization} we can remove the Rademacher variables \(\eps_i\) in this statement at the cost of changing the constants. This gives us that \(\E^\star\|n^{-1/2}\sum_{i=1}^n Z_i\|_{\calF_{\delta_n}}\to 0\) if and only if \(\E^\star\|n^{-1/2}\sum_{i=1}^n \xi_iZ_i\|_{\calF_{\delta_n}}\to 0\). By Lemma~\ref{lemma:convergence-characterizations} this is equivalent to asymptotic equicontuity and weak convergence.
\end{proof}
\begin{corollary}[Unconditional Multiplier Central Limit Theorem]
    \label{corr:unconditional-multiplier-clt}
	Let \(\calF\) be Donsker with \(\|P\|_\calF <\infty\). Let \(\xi_1,\dots,\xi_n\) be i.i.d random variables with mean \(\mu\), variance \(\sigma^2\) and \(\|\xi_i\|_{1,2} <\infty\) generated independently of \(X_1,\dots,X_n\). Then
	\[
		\frac{1}{\sqrt{n}}\sum_{i=1}^n \left(\xi_i\delta_{X_i}-\mu P\right)\wcov \mu\mathbb{G}+\sigma\mathbb{G}' +\sigma Z P
	.\] 
	where \(\mathbb{G}\) and \(\mathbb{G}'\) are independent (tight) processes on \(\ell^\infty(\calF)\) and are both independent of \(Z\sim N(0,1)\). The limiting process \(\mu\mathbb{G}+\sigma\mathbb{G}'+\sigma ZP\) is a mean zero Gaussian process\footnote{Mean Zero means that \((\mu\mathbb{G}+\sigma\mathbb{G}+\sigma ZP)f = 0\) for all \(f\in P\), Gaussian means that the marginals are normally distributed.} and covariance function \((\sigma^2+\mu^2)Pfg-\mu^2(Pf)(Pg)\).
\end{corollary}
We now want to show a conditional version of this result. That is, we want to show that we have weak convergence of the multiplier central limit theorem for almost all sequences \(Z_1,\dots,Z_n\).\footnote{And for this to be useful for bootstrap, we'd like the weak limit to be the same as the (non multiplier) empirical process.} For finite \(\calF\), this sort of result is a simple consequence of the Lindeberg central limit theorem. Before getting into it, it is useful to recall the central limit theorem as it applies to independent, but not necessarily identically distributed, data.

\begin{theorem}[Lindeberg Central Limit Theorem]
	\label{thm:lindeberg-clt}
	Suppose that \(X_1,X_2,\dots\) is a sequence of independent random vectors in \(\SR^k\)with \(\E[X_i] = \mu_i\). Define \(V_n = \frac{1}{n}\sum_{i=1}^n \Var(X_i)\) and suppose the following \emph{Lindeberg Condition} is satisfied:
	\begin{equation}
		\label{eq:lindeberg-condition}
		\tag{LC}
		\lim_{n\to\infty} \frac{1}{n}\sum_{i=1}^n \E\left[\left\|X_i-\mu_i\right\|^2\mathds{1}\{\|X_i-\mu_i\|>\eps \sqrt{n}\}\right] = 0,\;\;\;\;\;\;\forall \eps >0
	\end{equation}
	Then 
	\[
		V_n^{-1/2}\frac{1}{\sqrt{n}}\sum_{i=1}^n (X_i-\mu_i) \wcov N\left(0,I_k\right)
	.\] 
\end{theorem}

\begin{lemma}[Conditional Multiplier CLT for Finite Classes]
	\label{lemma:conitional-multiplier-clt-finite}
	Let \(Z_1,Z_2,\dots\) be i.i.d random vectors with \(\E Z_i =0\) and \(\E\|Z_i\|^2 <\infty\) independent of the i.i.d sequence \(\xi_1,\xi_2,\dots\) with \(\E\xi_i=0\) and \(\E\xi_i^2=1\). Then, conditionally on \(Z_1,Z_2,\dots,\)
	\[
		\frac{1}{\sqrt{n}}\sum_{i=1}^n \xi_iZ_i\wcov N\left(0, \Var(Z_1)\right)
	,\]
	for almost every sequence \(Z_1,Z_2,\dots\)
\end{lemma}
\begin{proof}
	Treating \(Z_1,Z_2,\dots\) as just a stream of constant vectors, by the Lindeberg Central Limit Theorem, the statement is true for every sequence \(Z_1,Z_2,\dots\) such that, for every \(\eps > 0\),
	\[
		\frac{1}{n} \sum_{i=1}^n Z_iZ_i'\to \Var(Z_1)\andbox \frac{1}{n}\sum_{i=1}^n \|Z_i\|^2\E_\xi\left[\xi_i^2\left\{|\xi_i|\|Z_i\| > \eps\sqrt{n}\right\}\right]
	.\]
	By Kolmogrov Strong Law of Large Numbers, the first statement is true for almost all all sequences \(\left\{Z_i\right\}_{i=1}^\infty\). A finite second moment \(\E\|Z_i\|^2 <\infty\), implies that \(\max_{1\leq i\leq n}\|Z_i\|/\sqrt{n}\to0\) for almost all sequences, which gives that the second statement holds for almost all sequences \(\{Z_i\}_{i=1}^\infty\). Under a probability measure, the intersection of two (measurable) sets with measure 1 also has measure 1.
\end{proof}

Lemma~\ref{lemma:conitional-multiplier-clt-finite} provides the weak convergence of marginals in the multiplier processes. What remains is to show some version of asymptotic equicontinuity (Theorem~\ref{thm:vdv1.5.6} or \ref{thm:vdv1.5.7}) to show weak convergence in \(\ell^\infty(\calF)\). 

Let \(\text{BL}_1\) be the set of all bounded Lipschitz functions. That is the set of all functions \(h:\ell^\infty(\calF)\to[0,1]\) such that \(|h(z_1)-h(z_2)\leq \|z_1-z_2\|_\calF\) for every \(z_1,z_2\). Using the bounded Lipschitz functions, we can define a metric (the bounded Lipschitz metric) between two distribtutions on a space \(\mathbb{D}\). 
\[
	d_{\text{BL}}(L_1,L_2) = \sup_{f\in \text{BL}_1} \bigg|\inf f\,dL_1 - \int f\,dL_2\bigg|
.\] 
It turns out that weak convergence is equivalent to convergence in the bounded Lipschitz metric. 
\begin{theorem}[Weak Convergence and the Bounded Lipschitz Metric]
	\label{thm:weak-convergence-bounded-lipschitz}
	Weak convergence of separable (\Cref{def:separable-process}) Borel probability measures on a metric space \(\mathbb{D}\) corresponds to convergence in a topology that is metrizable by the bounded Lipschitz metric.
\end{theorem}

The following theorems (presented without proof) gives conditions for convergence of the multiplier central limit theorem. First the other probability version is given under the same conditions as \Cref{thm:unconditional-multiplier-clt} then the almost sure version is given under only slightly stronger conditions. 

\begin{theorem}[Conditional Multiplier Central Limit Theorem]
	\label{thm:conditional-multiplier-clt-probability}
	Let \(\calF\) be a class of measurable functions. Let \(\xi_1,\dots,\xi_n\) be i.i.d random variables with mean zero, variance one and \(\|\xi\|_{2,1}<\infty\), independent of \(X_1,\dots,X_n\). Let \(\mathbb{G}_n'=n^{1/2}\sum_{i=1}^n \xi_i\left(\delta_{X_i}-p\right)\). Then the following assertions are equivalent:
	\begin{enumerate}
		\item \(\calF\) is Donsker;
		\item  \(\sup_{h\in\text{BL}_1}\left|\E_\xi h(\mathbb{G}_n')-\E h(\mathbb{G})\right|\to 0\) in outer probability and the sequence \(\mathbb{G}_n'\) is asymptotically measurable.
	\end{enumerate}
\end{theorem}

\begin{theorem}[Conditional Mulitplier Central Limit Theorem]
	\label{thm:conditional-multiplier-clt}
	Let \(\calF\) be a class of measurable functions. Let \(\xi_1,\dots,\xi_n\) be i,i.d random variables with mean zero, variance 1, and \(\|\xi\|_{2,1}<\infty\), independent of \(X_1,\dots,X_n\). Define the multiplier process \(\mathbb{G}_n'=n^{-1/2}\sum_{i=1}^n \xi_i(\delta_{X_i}-P)\). Then the following assertions are equivalent:
	\begin{enumerate}
		\item \(\calF\) is Donsker with \(\mathbb{G}_n\wcov\mathbb{G}\) and \(P^\star\|f-Pf\|_\calF^2<\infty\);
		\item \(\sup_{h\in\text{BL}_1}\left|\E_\xi h(\mathbb{G}_n')-\E h(\mathbb{G})\right|\to 0\) outer almost surely and the sequence \(\E_\xi h(\mathbb{G}_n')^\star - \E_\xi h(\mathbb{G})_\star\) converges to zero almost surely for every \(h\in \text{BL}_1\). 
	\end{enumerate}
	Here \(h(\mathbb{G}_n')^\star\) and \(h(\mathbb{G}_n')_\star\) denote measurable majorants and minorants with respect to \(\left(\xi_1,\dots,\xi_n,X_1,\dots,X_n\right)\) jointly.
\end{theorem}

\subsection{The Emprirical Bootstrap}%
\label{subsec:empirical-bootstrap}

\Cref{subsec:mclt} gives results that are useful for establishing the consistency of the multiplier bootstrap. We now quickly describe the empirical bootstrap and give some results. Let \(\P_n\) be the empirical measure of an i.i.d sample \(X_1,\dots,X_n\) from a probability measure \(P\). Given the sample values, let \(\hat X_1,\dots,\hat X_n\) be an i.i.d sample from \(\P_n\). The \emph{bootstrap empirical distribution} is the empirical measure \(\hat \P_n = n^{-1}\sum_{i=1}^n  \delta_{\hat X_i}\) and the \emph{bootstrap empirical process} \(\hat{\mathbb{G}}_n\) is given 
\[
	\hat{\mathbb{G}}_n = \sqrt{n}\left(\hat \P_n - \P_n\right) = \frac{1}{\sqrt{n}}\sum_{i=1}^n \left(M_{ni}-1\right)\delta_{X_i}
.\] 
where \(M_{ni}\) is the number of times that \(X_i\) is redrawn from the original samples. We can also define a bootstrap empirical process where we draw \(k\) bootstrap values, \(\hat X_1,\dots,\hat X_k\). The corresponding bootstrap empirical process is 
\[
	\hat{\mathbb{G}}_{n,k} = \sqrt{k}\left(\hat\P_k-\P_n\right)=\frac{1}{\sqrt{k}}\sum_{i=1}^n \left(M_{ki}-\frac{k}{n}\right)\delta_{X_i}
.\]
In either case, it is important that the vector \((M_{k1},\dots,M_{kn})\) is independent of \(X_1,\dots,X_n\), that is multinomially distributed with parameters \(k\) and probabilities \(\frac{1}{n},\dots,\frac{1}{n}\) for any random sample of size \(n\).

Let \(\E_M\) denote the expectation with respect to the distribution of \((M_{k1},\dots M_{kn})\). In light of \Cref{thm:weak-convergence-bounded-lipschitz} which equates weak convergence of a sequence of probability measures to convergence in the bounded Lipschitz metric, the following theorem establishes the consistency of the bootstrap. 
\begin{theorem}[Consistency of the Empirical Bootstrap]
	\label{thm:consistency-empirical-bootstrap}
	Let \(\calF\) be a Donsker class of measurable functions such that \(\calF_\delta\) is measurable for every \(\delta > 0\). Then
	\[
		\sup_{h\in\text{BL}_1}\left|\E_M h(\hat{\mathbb{G}}_{n,k_n}) - \E h(\mathbb{G})\right| \overset{P^\star}\to 0
	.\] 
	as \(n\to \infty\) for any sequence \(k_n\to \infty\). Furthermore the sequence \(\E_M h(\hat{\mathbb{G}}_{n,k_n})^\star - \E_M h(\hat{\mathbb{G}}_{n,k_n})_\star\) converges to zero in probability for every \(h\in \text{BL}_1\). If \(\P^\star\|f-Pf\|^2_\calF<\infty\), then the convergence is also outer almost surely.
\end{theorem}
