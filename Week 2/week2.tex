\section{Math Review}%
\label{sec:math-review}

\subsection{Vector Spaces and Norms}%

\begin{definition}[Vector Space]
	\label{def:vector-space}
	A vector space \(X\) is a set of elements with two operations, addition (+) and scalar multiplication  \((\cdot)\), and an additive identity \(\mathbf{0} \in X\) satisfying:
	 \begin{enumerate}
		 \item \( x + y = y+ x\) 
		 \item \((x+y)+z = x + (y+z)\)
		 \item \( \mathbf{0} + x = x,\, \forall x \in X\)
		 \item \(\alpha(x+y) = \alpha x + \alpha y\)
		 \item \((\alpha + \beta)x = \alpha x + \beta x\)
		 \item \((\alpha\beta)x = \alpha(\beta)x\)
		 \item \(0x = \mathbf{0}\) and  \(1x = x\) 
	\end{enumerate}
\end{definition}
Examples include \(\SR^K\) and  \(\calC[a,b]\), the set of all continuous functions from \([a,b] \to \SR\).

\begin{definition}[Norm]
	\label{def:norm}
	Let \(X\) be a vector space.  A norm is a functional,  \(\left\|\cdot\right\|:X \to \SR\) satisfying
	\begin{enumerate}
		\item \(\left\|x\right\|\geq 0,\,\forall x \in X\) and \(\|x\|=0\) if and only if \(x = \mathbf{0}\).
		\item \(\left\|x+y\right\|\leq \|x\| + \|y\|\) \. (Triangle Inequality) 
		\item \(\|\alpha x\| = |\alpha|\|x\|,\,\forall \alpha \in \SR, x \in X\)
	\end{enumerate}
\end{definition}
Examples of norms include the \(\ell^p\) norms on  \(\SR^K\) or the sup-norm on the space of all bounded, real valued, functions. On \(\SR^K\) all norms are equivalent, which is to say that for any two norms  \(\|\cdot\|_1, \|\cdot\|_2\) there are constants \(C_1, C_2\) such that  \(C_1\|\cdot\|_2 \leq \|\cdot\|_1 \leq C_2 \|\cdot\|_2\). However, this is not generally the case for functional vector spaces. For example on \(\calC[a,b]\) there is no constant \(c\) such that, for all \(f\):
 \[
	 \sup_{x\in[a,b]}f(x) = \|f\|_\infty \leq c \|f\|_2 = \left(\int_{a}^{b} f^2(x) dx\right)^{1/2}
.\] 

Closely related to a norm is the concept of a metric, which is a way of defining a distance on a space. 
\begin{definition}[Metric]
	\label{def:metric}
	Let \(X\) be a vector space. A metric (or distance metric) on \(X\) is a functional  \(d(x,y): X \times X \to \SR\) satisfying:
	 \begin{enumerate}
		 \item \(d(x,y)\geq 0,\forall x,y\) and \(d(x,y) = 0 \iff x = y\)
		 \item \(d(x,y) = d(y,x)\)
		 \item \(d(x,y) \leq d(x,z) + d(z,y),\,\forall x,y,z\)
	\end{enumerate}
\end{definition}

It is straightforward to verify that, given a norm on a vector space \(X\), we can generate a valid metric:
 \[
	 d_{\|\cdot\|}(x,y) := \|x-y\|
.\]

We return to these concepts when discussing a topology. 

\subsection{Topology and Continuity}%
\label{subsec:topology}

A topology is a general structure under which we can discuss concepts such as convergence and continuity. We can start with a general structure and then discuss spaces where the topology is generated by a metric (or norm).

\begin{definition}[Topology]
	\label{def:topology}
	A topology on a set \(X\) is a collection of subsets of X, \(\boldsymbol{\tau} \subset 2^X\) satisfying:
	\begin{enumerate}
		\item \(\emptyset, X \in \boldsymbol{\tau}\).
		\item \(\boldsymbol{\tau}\) is closed under finite intersections, if \(\left\{A_k\right\}_{k=1}^K \in \boldsymbol{\tau}\) then \(\bigcap_{k=1}^K A_k \in \boldsymbol{\tau}\).
		\item \(\boldsymbol{\tau}\) is closed under arbitrary unions. For any index set \(I\), if \(\left\{A_k\right\}_{k\in I} \in \boldsymbol{\tau}\) then \(\bigcup_{k\in I} A_k \in \boldsymbol{\tau}\).
	\end{enumerate}
	The elements of \(A\in\boldsymbol\tau\) are called open sets. A set, \(B\), is closed if it's complement is in \(\boldsymbol{\tau}\), \(B^c \in \boldsymbol{\tau}\).
\end{definition}
Some simple examples include the trivial topology, \(\boldsymbol{\tau} = \left\{X, \emptyset\right\}\) and the discrete topology  \(\boldsymbol{\tau} = 2^\calX\). Given a topology, we can define some familiar terms:

\begin{definition}[Interior]
	\label{def:interior}
	For a subset \(A \subseteq X\), the interior of  \(A\), denoted \(A^\circ\), is the largest open set included in A (where largest is defined under the usual subset ordering). We can also express this as the union of all open sets contained by \(A\).
	 \[
		 A^\circ = \bigcup\left\{B: B \in \boldsymbol{\tau}, B \subseteq A\right\}
	.\] 
\end{definition}
Note that a set is open if and only if \(A = A^\circ\).

\begin{definition}[Closure]
	\label{def:closure}
	For a subset \(A \subseteq X\), the closure of  \(A\), denoted  \(\bar{A}\), is the smallest closed set the covers A. We can express this as the intersection of all closed sets containing A:
	 \[
		 \bar{A} = \bigcap \left\{B: B^c \in \boldsymbol{\tau}, A \subseteq B\right\}
	.\] 
\end{definition}
By De-Morgan's law and closure of the topology under arbitrary union we can see that this intersection always gives a closed set. A set is closed if and only if \(A = \bar{A}\).

\begin{lemma}[]
	\label{lemma:prelim}
	Suppose \(x \in \bar{A}\), then for every neighborhood of  \(x\),  \(V_x\), we have that  \(V_x \cap A \neq \emptyset\).
\end{lemma}
\begin{proof}
	Let \(x \in \bar{A}\) and suppose for some neighborhood  \(V_x\) of \(x\) we have that  \(V_x \cap A = \emptyset\). Then we know that  \(V_x^\circ \cap A = \emptyset\). Take  \(\tilde A = \bar A \cap (V_x^\circ)^c\). We can verify that this is a smaller closed set that also contains  \(A\).
\end{proof}


\begin{definition}[Boundary]
	\label{def:boundary}
	The boundary of a set \(A\), denoted  \(\delta A\), is  \(\bar{A}\setminus A^\circ\).
\end{definition}

A useful concept when talking about convergence under a topology is that of a neighborhood of a point \(x \in X\).

\begin{definition}[Neighborhood]
	\label{def:neighborhood}
	For a point \(x \in X\) a set  \(V\) is a neighborhood of  \(X\) if  \(x \in V^\circ\).
\end{definition}

We can now use the topology to define limit points and convergence.

\begin{definition}[Limit Point]
	\label{def:limit-point}
	A point \(x \in X\) is a limit point of a set  \(A \subseteq X\) if, for every neighborhood  \(V\) of  \(x\), 
	 \[
		 A \bigcap\left( V\setminus\{x\} \right)\neq \emptyset
	.\] 
	In other words, every neighborhood of \(x\) intersects with  \(A\) at a point other than \(x\). Let \(A'\) be the set of all limit points of  \(A \subseteq X\).
\end{definition}
\begin{lemma}[]
	\label{lemma:prelim2}
	If \(S\) is a subset of  \(X\), then  \(\bar{S} = S \cup S'\).
\end{lemma}
\begin{proof}
	First show that \(\bar{S} \subseteq S \cup S'\). Let \(x \in \bar{S}\). If \(x \in S\) then we are done. Otherwise, suppose \(x \in \bar{S}\setminus S\). This means that for all \(V_x\) we have that \(S \cap V_x = S\cap \left(V_x\setminus\{x\}\right).\) By the result of Lemma~\ref{lemma:prelim}, we have that \(V_x \cap S \neq \emptyset\). So, \(x \in S'\). 

	Now suppose that \(x \in S \cup S'\). Clearly if  \(x \in S\) then \(x\in \bar{S}\). Suppose then that  \(x \in S'\setminus S\) but \(x \not\in \bar{S}\). Let \(\tilde S\) be any closed set containing  \(S\), that is  \(S \subseteq \tilde S\). For sake of contradiction, suppose that  \(x \not\in \tilde S\) (x is a limit point of \(S\) that is not in  \(\tilde S\)). Because \(\tilde S\) is closed we know that  \(\tilde S^c \in \boldsymbol{\tau}\). Further, we know that \(x \in \tilde S^c\) so that  \(\tilde S^c\) is a neighborhood of \(x\). Since \(x\) is a limit point of  \(S\), we know that  \(\tilde S^c \cap S = \tilde S^c \cap S \setminus\{x\} \neq \emptyset\). However, we also know that \(S \subseteq \tilde S\) so we have a contradiction. Therefore, it must be that  \(x \in \bar{S}\) which completes the proof.
\end{proof}


\begin{lemma}[Characterization of Closed Sets]
	\label{lemma:closed}
	A set is closed if and only if it contains all of its limit points.
\end{lemma}
\begin{proof}
	This is a consequence of Lemma~\ref{lemma:prelim2} and the fact that \(A\) is closed if and only if  \(\bar{A} = A\).
\end{proof}


\begin{definition}[Convergence]
	\label{def:convergence}
	We say a sequence \(\left\{x_n\right\}_{n=1}^\infty\) converges to a point \(x \in X\) if for every neighborhood  \(V_x\) of \(x\), there exists a number  \(M\) such that for all  \(m\geq M\), \(x_m \in V_x\). 
\end{definition}

Note that under the trivial topology \(\boldsymbol{\tau} = \left\{\emptyset, X\right\}\) all sequences converge to any point \(x \in X\) whereas under the discrete topology on  \(\SR\),  \(\boldsymbol{\tau} = 2^\SR\), no sequence converges.


\begin{definition}[Continuity]
	\label{def:continuity}
	Let \((\calX, \boldsymbol{\tau}_1)\) and \((\calY, \boldsymbol{\tau}_2)\) be two topological spaces and \(f:\calX\to\calY\). We say  \(f\) is continuous if  \(f^{-1}(A) \in \boldsymbol{\tau}_1\) for all \(A \in \boldsymbol{\tau}_2\). That is, a continuous function maps open sets to open sets. 
\end{definition}

We can now get ready to combine the notions of continuity and convergence coming from a topology with the notions that we are familiar with from metric spaces. First, we need to define the topology generated by a metric. 

\begin{definition}[Generated Topology]
	\label{def:generated}
	Let \(\calA\) be a collection of subsets of  \(X\). The topology generated by  \(\calA\), \(\langle\calA\rangle\) is the smallest topology that contains \(\calA\):
	\[
		\langle\calA\rangle = \bigcap \left\{\boldsymbol\tau: \calA\subseteq\boldsymbol{\tau}\right\}
	.\] 
\end{definition}

We will then define the topology generated by a metric as the topology generated by the collection of open balls \(B(x,\eps)\).

\begin{definition}[Open Ball]
	\label{def:open-ball}
	Let \(d(x,y)\) be a metric on  a vector space  \(X\). For any point  \(x \in X\) define the open ball of size  \(\eps\) around  \(x\) as:
	\[
		B(x,\eps) = \left\{y: d(x,y) \leq \eps\right\}
	.\] 
\end{definition}

In a metric space, we consider the topology generated by all the open balls \(\boldsymbol{\tau}_d = \langle\left\{B(x,\eps):x \in X, \eps > 0\right\}\rangle\). In fact, the set of open balls is a basis for this topology, which means that every open set \(A\) in \(\boldsymbol{\tau}_d\) and any point \(x \in A\), there is an open ball \(B\) such that  \(x \in B \subseteq A\).\footnote{In fact, the set of all open balls with rational \(\eps\) is a basis for the topology}. Many topological properties such as continuity or convergence can be verified by simply confirming the properties for all members of a basis for the topology. This ties together the ``epsilon-delta" notions of continuity and convergence with the more topological versions given above.

For the rest of this subsection we will talk about separability and compactness, but give examples using normed-metric spaces instead of talking in generality about the topology.

\begin{definition}[Dense Subset]
	\label{def:dense}
	A topological space \((X,\boldsymbol{\tau})\) has a dense subset \(\calA\) if  \(\bar{\calA} = X\). Equivalently, by Lemma~\ref{lemma:prelim2}, every point of  \(X\) is either in  \(\calA\) or is a limit point of  \(\calA\).
\end{definition}
Informally, all points in \(X\) are either in \(\calA\) or arbitrarily ``close" to \(\calA\). As an example, in the standard topology on \(\SR\) generated by the metric \(d(x,y) = |x-y|\), the rationals \(\SQ\) are dense. We also have that, for the set of continuous functions under the sup norm, the set of all polynomials is dense, which means that we can approximate a function arbitrarily well with them.

\begin{definition}[Seperable Space]
	\label{def:seperable}
	We say that a topological space \((X,\boldsymbol{\tau})\) is separable if it has a countable dense subset.
\end{definition}
As we went over above, the real line with its standard topology is separable. The \(L_p[a,b]\) spaces are also generally separable for  \(1 \leq p \leq \infty\). However \(L_\infty\) is not separable, which will cause issues (this is not the example below).


\begin{example}[Bounded functions with the sup norm is not seperable]
	\label{ex:notseperable}
	Let \(\{f_i\}_{i\in\SN}\) be a countable set of functions on \(B_\infty[a,b]\). Let  \(\{q_i\}_{i\in\SN}\) be some counting of the rational numbers between a and b. Let \(\tilde f\) be some function that is equal to 0 except on the rational numbers. For each rational number  \(q_i\) define 
	 \[
		 \tilde{f}(q_i) = \begin{cases}
			 1 & \text{if }f_i(q_i) \leq 0 \\
			 -1 & \text{if }f_i(q_i) > 0
	    \end{cases}
	.\] 
	We can see that \(\tilde{f}\) is bounded (and integrates to 0), but it is at least distance one from each function in \(\{f_i\}_{i\in\SN}\).
\end{example}

Initially I thought this example would work for \(L_\infty[a,b]\), but this only forces a difference on a set of measure 0 and I believe \(L_\infty\) works with an essential supremum norm.

Another important/useful concept is that of compactness. The general notion is given below:

\begin{definition}[Compact Set]
	\label{def:compact}
	A set \(A\) is compact if for every collection of open sets  \(\{G_i\}\) such that \(A\subset \bigcup G_i \), there is a finite subcollection that also covers \(A\).
\end{definition}
\begin{example}[]
	\label{ex:reals}
	The real-line is not compact. Consider the open cover \(\left\{(n,n+1)|n\in\SZ\right\} \)
\end{example}
\begin{example}[]
	\label{ex:halfopen}
	The interval \((0,1]\) is not compact. Consider the open cover  \(\left\{(1/n, 1+1/n)| n\in \SN \right\}\)
\end{example}
\begin{theorem}[Heine-Borel]
	\label{thm:heine-borel}
	For a subset \(S\) of the Euclidean Space\footnote{That is the space \(\SR^n\) equipped by the topology generated by the standard distance metric}, \(\SR^n\), the following statements are equivalent:
	 \begin{itemize}
		\item \(S\) is closed and bounded
		\item  \(S\) is compact
	\end{itemize}
\end{theorem}
Compactness is nice because of various extreme value theorems that ensure that a supremum or infimum is attained. Heine-Borel gives a nice way of characterizing compactness for Euclidean Spaces, but in general there is no equivalent result for general metric spaces. We have to strengthen the boundedness assumption.

\begin{definition}[Totally Bounded]
	\label{def:totally-bounded}
	A set \(\calA\) is totally bounded if for each  \(\eps> 0\) there exists a finite sequence  \(\{a_1,\dots,a_n\}\) such that for \(B_i = \left\{a\in\calA: \left\|a-a_i\right\|\leq \eps\right\}\), \(\bigcup_{i=1}^N B_i\) covers  \(A\).	
\end{definition}

\textbf{Intuition:} For any precision \(\eps\), you can find a finite set of points that describe  \(\calA\) arbitrarily well. (much more demanding in infinite dimensions than just bounded).

\begin{theorem}[]
	In a complete metric space, the following are equivalent:
	\begin{itemize}
		\item \(\calA\) is a compact subset
		\item \(\calA\) is closed and totally bounded 
 		\item Every sequence in \(\calA\) has a convergent subsequence which converges to a point in  \(\calA\).
	\end{itemize}
\end{theorem}

For a compact set \(T\), let  \(C(T)\) be the set of continuous functions from  \(T\) to  \(\SR\) equipped with the sup norm. We may want to characterize when a subset \(K\) of  \(C(T)\) is compact. 

\begin{definition}[Equicontinuous]
	\label{def:equicontinuous}
	A set of functions \(K \subseteq C(T)\) is equicontinuous if for every  \(t_0 \in T\) and  \(\eps > 0\) there is a  \(\delta > 0\) such that  \(|f(t)-f(t_0)|<\eps\) whenever  \(\|t-t_0\|<\delta\) \textbf{for all} \(f\in K\).
\end{definition}
This is a bit like to uniformly continuity but adapted a bit to deal with a function space. 
\begin{theorem}[Arzela-Ascoli]
	\label{thm:aa}
	If  \(T\) is compact, then  \(K \subseteq C(T)\) is compact (under the sup-norm) if and only if  \(K\) is bounded and equicontinuous.
\end{theorem}

This concludes our discussion of topology and continuity. We now review measurability.

\subsection{Probability Spaces and Outer Measure}%
\label{sec:probability}
\begin{definition}[Sigma Algebra]
	\label{def:sigma-algebra}
	A collection of subsets \(\calF\) is a sigma-algebra (or sigma-field) if it contains the whole set and is closed under complement and under countable union.
\end{definition}

\begin{definition}[Borel Sigma Algebra]
	\label{def:borel}
	For any collection of sets \(\calA\), we call the smallest sigma algebra containing \(\calA\), \(\sigma(\calA)\), the sigma algebra generated by \(\calA\). The Borel sigma algebra  on a topological space is the sigma algebra generated by all the open sets, \(\calB(X) = \sigma(\boldsymbol{\tau})\).
\end{definition}

The Borel sigma algebra is useful as it makes all continuous functions measurable (defined below). 

\begin{definition}[Probability Space]
	\label{def:prob-space}
	A probability space is a triple \((\Omega, \calF, \P)\) consisting of a set of elements \(\Omega\), a sigma algebra on  \(\Omega\),  \(\calF\), and a probability measure  \(\P:\calF \to [0,1]\) satisfying:
	 \begin{enumerate}
		 \item \(\P(A) \geq \P(\emptyset) = 0\) [Non-negativity]
		 \item If \(A_i \in \calF\) is a countable sequence of disjoint sets then  \(\P\left(\bigcup_i A_i\right) = \sum_{i} \P(A_i)\) 
		 \item \(\P(\Omega) = 1\).
	\end{enumerate}
\end{definition}

A measurable function between two spaces equipped with sigma algebra's is simply one that maps measurable sets to measurable sets, similar to the definition of a continuous function. 

\begin{definition}[Measurable Map]
	A function  \(f:(\calX, \calA) \to (\calY, \calB)\) is measurable if  \(f^{-1}(B)\in\calA\) for all  \(B \in \calB\)	
\end{definition}

\begin{lemma}[Lemma 1.3.1 VdV\& W]
	\label{lemma:vdv1.3.1}
	The Borel \(\sigma\)-field on a metric space \(\mathbb{D}\) is the smallest  \(\sigma\)-field that makes all elements of \(C_b(\mathbb{D})\) measurable (with respect to the Borel sets on  \(\SR\)).\footnote{\(C_b(\mathbb{D})\) is the set of all continuous bounded functions from  \(\mathbb{D}\to\SR\), where  \(\SR\) is endowed with the standard topology on the real line}.
\end{lemma}
\begin{proof}
	For any closed set \(F\), \(F\) is the null set  \(\left\{x:f(x)=0\right\}\) of the continuous, bounded function, \(x \mapsto d(x,F)\land 1\). Since the singleton  \(\{0\}\) is a closed set in  \(\SR\) (all metric spaces are Hausdorff),  \(F\) must be in the sigma algebra on  \(\mathbb{D}\) to make  \(d(x,F)\land 1\) measurable. Since all the closed sets generate the Borel  \(\sigma\)-field (because \(\sigma\)-fields are closed under complement), all Borel sets must be included in the sigma-algebra on \(\mathbb{D}\).
\end{proof}


Given this, we can abstractly think about a random variable as a measurable map from a probability space into another measurable space (typically the real-line). Measurability ensures that things like expectations and probabilities of random variables are well defined.

However, measurability becomes a problem when we are dealing with random functions. For example, if \(X\) is a map from a probability space to  \(L_\infty[a,b]\), the Borel-sigma algebra on  \(L_\infty[a,b]\) is quite large (its not separable). This means that measurable sets in  \(L_\infty[a,b]\) may not map back to measurable sets on the probability space \(\Omega, \calF, \P\). 

This is a problem because \(L_\infty\) is typically a useful space to work in for empirical process theory. So we have to find a way to relax measurability. This means that we work with outer expectations and probabilities:
\begin{definition}[Outer Measure and Inner Measure]
	Let \((\Omega,\calF,\P)\) be a probability space \(T:\Omega \to \SR\). Define the outer expectation:
	 \[
		 \E^\star[T] = \inf\left\{\E[U]: T\leq U, U\text{ is measurable}\right\}
	.\]
	and the inner expectation:
	\[
		\E_\star[T] = \sup\left\{\E[U]: U\leq T, U\text{ is measurable}\right\}
	.\] 
\end{definition}
We can use this to define inner and outer probability measures by restricting \(T\) to be the indicator function for an arbitrary set  \(B\). Inner and outer expectations are generally nicely behaved but they require modified versions of dominated and monotone convergence and Fubini's theorem breaks down.

\section{Weak Convergence}%
\label{sec:weak-convergence}

\subsection{Definition and Characterizations}%
\label{subsec:wcov-def}

We can now talk about weak convergence of random variables. Let \(X_n\) be a real-valued random variable with cdf  \(F_n(t)\) and let  \(X\) be a random variable with cdf  \(F(t)\). The typical definition of weak convergence is that \(X_n \overset{L}{\to} X\) if  \(F_n(t) \to F(t)\) pointwise at all continuity points of  \(F\). This is not super general for non-real valued random maps.

\begin{theorem}[Portmanteau]
	\label{thm:portmanteau}
	For real random variables \(X_n\overset{L}{\to}X\) is equivalent to:
	 \begin{itemize}
		 \item \(\E\left[g(X_n)\right] \to \E[g(X)]\) for all bounded continuous functions. 
		 \item For all open sets \(G\),  \(\lim\inf \P(X_n\in G) \geq P(X\in G)\).
		 \item For all closed sets \(K\),  \(\lim\sup \P(X_n \in K) \leq \P(X \in K)\).
	\end{itemize}
\end{theorem}

This motivates the theory of weak convergence for general metric spaces. Let \(\mathbb{D}\) be a complete metric space with metric  \(d\). We can equip \(\mathbb{D}\) with it's Borel-sigma algebra as defined in Definition~\ref{def:borel} and a tight probability measure as defined in Definition~\ref{def:tight}. Let  \(C_b\left(\mathbb{D}\right)\) be the set of all continuous and bounded real functions on \(\mathbb{D}\). If \(X\) is a random variable,  \(X:(\Omega,\calF,\P) \to \mathbb{D}\) then it's law is given \(L = \P\circ X^{-1}\).


\begin{definition}[Tight Probability Measure]
	\label{def:tight}
	A probability measure is tight if for every \(\eps > 0\) there is a compact set  \(K_\eps\) such that  \(P(K_\eps) \geq  1-\eps\)
\end{definition}
This is a generalization of bounded in probability I believe. 
\begin{definition}[Borel Law]
	\label{def:borel-law}
	For a random variable \(X\), we say that  \(X\) has a Borel Law  \(L\) if
	 \[
		 \P\left(X\in A\right) = \int_A dL
	.\] 
	for all Borel sets \(A\). 
\end{definition}
Given this setup, we can now define weak convergence:
\begin{definition}[Weak Convergence]
	\label{def:wcov}
	Let \((\Omega_n, \calF_n,\P_n)\) be a sequence of probability spaces and  \(X_n:\Omega_n \to \mathbb{D}\). Then we say that \(X_n\overset{L}{\to}X\) if:
	 \[
		 \E^\star\left[f(X_n)\right]\to \E[f(X)]
	.\] 
	for every \(f \in C_b(\mathbb{D})\)
\end{definition}

We can characterize this convergence using another Portmanteau theorem. 
\begin{theorem}[Portmanteau]
	\label{thm:portmanteau2}
	The following are equivalent:
	\begin{enumerate}
		\item \(X_n \overset{L}{\to} X\)
		\item \(\lim\inf \P_\star(X_n \in G) \geq \P(X \in G)\) for all open sets \(G\).
		\item \(\lim\sup \P^\star(X_n \in F) \leq \P(X \in F)\) for every closed set \(F\).
		\item \(\lim P(X_n \in B) = P(X \in B)\) for every Borel set  \(B\) with  \(P(X \in \delta B) = 0\).
	\end{enumerate}
\end{theorem}
\textbf{Question:} Is \(X\) supposed to have a Borel Law? Otherwise where do open and closed sets get tied into this? Is it from the notion of convergence?
\begin{proof}
	This proof is in a few steps. 
	
	(4)\(\implies\)(3): Suppose that \(\lim P(X_n\in B)= P(X\in B)\) for every Borel set  \(B\) with  \(\P(X\in\delta B) = 0\). Let \(F\) be a closed set and let  \(F^\eps = \{x: d(x,F)<\eps\}.\) The sets \(\delta F^\eps\) are disjoint for different values of  \(\eps > 0\) (The boundary of this set is \(\delta F^\eps = \{x: d(x,F) = \eps\} \)), so at most countably many of them can have nonzero L-measure (otherwise the measure of the entire space would be infinite). Choose a sequence \(\eps_m \downarrow 0\) with  \(L(\delta F^{\eps_m})=0\) for each  \(m\) (this is possible because only countably many \(\eps\) have \(L(F^\eps)\neq 0\)). For a fixed \(m\), by (4) we have that:
	 \[
		 \lim\sup P^\star\left(X_\alpha \in F\right)\leq \lim\sup P^\star\left(X_\alpha \in \overline{F^{\eps_m}}\right) = L\left(\overline{F^{\eps_m}}\right)
	.\]
	letting \(m\to\infty\) gives (3).

	(3)\(\iff\)(2): Take any closed set \(F\). Its complement  \(F^c\) is open. If 
	 \[
		 \lim\inf \P_\star(X_n \in F^c) \geq \P(X \in F^c)
	.\] 
	Then
	\begin{align*}
		\lim\sup \P^\star(X_n \in F) &\leq \lim\inf 1 - \P_\star(X_n \in F^c) \\ 
									   &\leq 1 - \P(X \in F^c)\\
									   &= \P(X \in F)
	\end{align*}
	a symmetric argument shows the backwards direction.
	
	(2)+(3)\(\implies\)(4): This is straightforward if we recall that, for any set with \(L(\delta B)= 0\) we have that \(L(B) = L(\bar{B})\). Then we bound the  \(\limsup\) by the  \(\liminf\):
	 \[
		 \lim\sup \P^\star(X\in B)\leq \lim\sup\P(X \in \bar{B}) \leq \P(X \in \bar{B}) = \P(X\in B) \leq \lim\inf \P_\star(X_n \in B)
	.\] 
	which gives (4).

	(1)\(\implies\)(2): Take any  \(G\) open and define the sequence of functions:
	\begin{equation*}
		f_m(x) := \min(1, m\cdot d(x, G^c))
	\end{equation*}
	Notice that \(f_m(x) \in C_b(\mathbb{D})\) and \(f_m(x) \leq \mathds{1}\{x \in G\}\). So, for every \(m\) we have that
	\begin{align*}
		\lim\inf \P_\star(X\in G) &= \lim\inf \E_\star\left[\mathds{1}\{X\in G\}\right] \\
								  &\geq \lim\inf \E_\star\left[f_m(X) \right]\\
								  &\geq \E[f_m(X)]
	\end{align*}
	since \(f_m(x)\uparrow \mathds{1}\{X\in G\}\) by monotone convergence we get the result in (2).

	\textbf{Question:} How do we know from weak convergence that this sequence converges in inner expectation? 

	By VdV and Wellner, weak convergence implies (is equivalent to) \(\lim\inf \E_\star\left[f(X_n)\right] \geq \E\left[f(X)\right]\) for every bounded, Lipschitz continuous, non-negative \(f\). I think the argument for why this is the case goes: Let \(f \geq 0\) be bounded and continuous. Then by weak convergence
	\[
		\lim\sup \E^\star[-f(X_n)] = \E[-f(X)]
	.\] 
	Taking negatives will give:
	\[
		\lim\inf \E_\star[f(X_n)] \geq - \lim\sup \E^\star[-f(X_n)] = \E[f(X)]
	.\] 
	In any case, \(f_m(X)\) is Lipschitz continuous which gives the result. 

	(2)\(\implies\)(1): (SKETCH)
	\begin{itemize}
		\item Suppose \(f(x) \geq 0\) is continuous and bounded
		\item Approximate it from above and below by indicator functions of open sets.
	\end{itemize} 
\end{proof}

Weak convergence is nice because it gives the continuous mapping theorem. 
\begin{theorem}[Continuous Mapping Theorem]
	\label{thm:cmt}
	Let \(g:\mathbb{D}\to \mathbb{E}\) be continuous at every point \(\mathbb{D}_0 \subseteq \mathbb{D}\). If  \(X_n \overset{L}\to X\) and  \(\P(X\in\mathbb{D}_0) = 0\) then  \(g(X_n) \overset{L}\to g(X)\).
\end{theorem}

\begin{proof}
	(Without Discontinuity Points):
	Let \(Z_n = g(X_n)\) and  \(Z = g(X)\). We want to show that  \(\E^\star\left[f(Z_n)\right] \to \E\left[f(Z)\right]\) for all \(f \in C_b(\mathbb{D};\mathbb{E})\).
	\[
		\lim_{n\to\infty} \E[f(Z_n)] = \lim_{n\to\infty}\E[f(g(X_n))] = \E[f(g(X))] = \E[f(Z)]
	.\]
	The main step here is weak convergence of \(X_n\) and the stability of  \(C_b(\mathbb{D};\mathbb{E})\) under composition.

	(With Discontinuity Points, from VdV\&W):
	The set \(D_g\) of all points at which  \(g\) is discontinuous can be written
	\[
		D_g = \bigcup_{m=1}^\infty\bigcap_{k=1}^\infty \left\{x : \exists y,z \in B(x,1/k) \text{ with }d_\mathbb{E}(g(y),g(x)) > 1/m\right\}
	.\] 
	\textit{Intuition: Recall that \(g\) is continuous at  \(x\) if for every  \(m \in \mathbb{N}\) there exists a  \(k \in \SN\) such that\footnote{Topologically, this is saying that the inverse map of every open neighborhood of \(f(x)\) is an open neighborhood of  \(x\)}  \[y,z \in B(x,1/k)\implies d_\mathbb{E}(g(y),g(z)) < 1/m\] If the function is not continuous at \(x\) you can find a counterexample for some  \(k,m\in\SN\).}
	
	Let \(G_k^m = \left\{x : \exists y,z \in B(x,1/k) \text{ with }d_\mathbb{E}(g(y),g(x)) > 1/m\right\}\). Every \(G_k^m\) is open (if \(x\) is in  \(G_m^k\) the points just around  \(x\) will be as well so that we can write \(G_m^k\) as a union of open balls) so that \(D_g\) is a Borel set. For every closed  \(F\) we then have that:
	 \[
		 \overline{g^{-1}(F)} \subseteq g^{-1}(F)\cup D_g
	.\]
	By Portmanteau:
	\begin{align*}
		\lim\sup \P^\star\left(g(X_n)\in F\right) \leq  \lim \sup P^\star\left(X_n \in \overline{g^{-1}(F)}\right) &\leq \P\left(X \in \overline{g^{-1}(X)}\right) \\ &= \P\left(X \in g^{-1}(F)\right) \\ &= \P\left(g(X) \in F\right)
	\end{align*}
	Applying Portmanteau again gives weak convergence.
\end{proof}
\begin{example}[]
	\label{ex:uniform}
	Take \(\mathbb{G}_n \in L^\infty(\SR)\): 
	\[\mathbb{G}_n(t) := \frac{1}{\sqrt{n}}\sum_{i=1}^n\left( \mathds{1}\{X_i \leq t\} - \E\left[\mathds{1}\{X \leq t\}\right]\right)\]
	and suppose that \(\mathbb{G}_n \overset{L}{\to} \mathbb{G}\) where \(\mathbb{G}\) is some other element of  \(L^\infty(\SR)\). Let  \(Z: L^\infty(\SR) \to \SR\) be defined as:
	 \[
		 Z(f) := \sup_{t} |f(t)|
	.\]
	this function is continuous. Applying the continuous mapping theorem to \(Z\) allows us to build uniform confidence intervals. 

	Let \(\gamma_{1-\alpha}\) be the  \(1-\alpha\) quantile of  \(Z := \sup_t |\mathbb{G}(t)|\) and construct a confidence interval (at each \(t\)):
	\[
		\left[\frac{1}{n}\sum_{i=1}^n\mathds{1}\{X_i\leq t\}  -\gamma_{1-\alpha}/\sqrt{n}\,,\,\frac{1}{n}\sum_{i=1}^n\mathds{1}\{X_i\leq t\}+\gamma_{1-\alpha}/\sqrt{n}\right]
	.\]
	Then:
	\begin{align*}
		&\P\left(\frac{1}{n}\sum_{i=1}^n\mathds{1}\{X_i \leq t\}  -\gamma_{1-\alpha}/\sqrt{n} \leq \E\left[\mathds{1}\{X\leq t\}\right] \leq \frac{1}{n}\sum_{i=1}^n \mathds{1}\{X_i \leq t\} + \gamma_{1-\alpha} /\sqrt{n}\;:\;\text{ for all } t \right) \\
		&= \P\left(\left|\mathbb{G}_n(t)\right| \leq \gamma_{1-\alpha}\;\forall t\right) \\
		&= \P\left(\sup_t |\mathbb{G}_n(t)| \leq \gamma_{1-\alpha}\right)
	\end{align*}
	But by continuous mapping theorem and Portmanteau, if \(\P(\sup_t\left|\mathbb{G}\right|=\gamma_{1-\alpha}) = 0\):
	\[
		\lim_{n\to\infty}\P\left(\sup_t\left|\mathbb{G}_n(t)\right|\leq \gamma_{1-\alpha}\right) = \P\left(\sup_t \left|\mathbb{G}(t)\right|\leq \gamma_{1-\alpha}\right) = 1-\alpha
	.\]
	This sort of argument can be applied more generally to functions \(\mathbb{G}_n(t) = \hat{m}(t)-m(t)\) to construct uniform confidence intervals.
\end{example}

This shows the usefulness of Portmanteau and Continuous Mapping Theorem. For finite dimension vectors we can use the central limit theorem to establish weak convergence to a normal distribution. However, when \(X_n\) is a random element in  \(L^\infty\) it may be harder to show that \(X_n \wcov X\) for some other  \(X \in L^\infty\).
 \begin{itemize}
	 \item Don't want to check \(\E[f(X_n)] \to \E[f(X)]\) for all  \(f\in C_b(L^\infty)\) [There are at least 20 functions in this class]
\end{itemize}
Instead we will try to use the structure of \(L^\infty\) to show the result. 

 \begin{definition}[Asymptotic Tightness]
	\label{def:asymptotically-tight}
	A sequence \(X_n\) of random maps is asymptotically tight if for every  \(\eps,\delta>0\) there is a compact \(K_\eps\) such that 
	 \[
	    \lim\inf P_\star\left(X_n \in K_\eps^\delta\right) \geq 0
	.\]
	where \(K_\eps^\delta = \{y\in\mathbb{D}:d(y,K_\eps) < \delta\}\) is the ``\(\delta\)-enlargement" around \(K_\eps\). 
\end{definition}
\begin{definition}[Asymptotic Measurability]
	\label{def:asymptotically-measurable}
	A sequence \(X_n\) of random maps is asymptotically measurable if for all  \(f \in C_b(\mathbb{D})\) :
	\[
		\E^\star f(X_n) - \E_\star f(X_n) \to 0
	.\] 
\end{definition}

We would like for a sequence \(X_n\) that weakly converges to an element  \(X\) to inherit some properties from  \(X\):

\begin{lemma}[Lemma 1.3.8 VdV\& W]
	\label{lemma:vdv1.3.8}
	The following are true:
	\begin{enumerate}
		\item If \(X_n \overset{L}{\to} X\) then \(X_n\) is asymptotically measurable
		\item If \(X_n \overset{L}{\to} X\) then \(X_n\) is asymptotically tight if and only if \(X\) is tight.
	\end{enumerate}
\end{lemma}
\begin{proof}
	(1): Take any function \(f \in C_b(\mathbb{D})\). By definition of weak convergence we know that 
	 \[
		 \lim\E^\star\left[f(X_n)\right] =\E[f(X)]\andbox \lim\E^\star\left[-f(X_n)\right]= \E[-f(X_n)]
	.\] 
	I think we should have that \(-\E_\star\left[f(X_n)\right] \geq \E^\star\left[-f(X_n)\right]\) for any \(f\) which give the result (I think this holds with equality but I leave it as an inequality since this is all we need for the result).

	(2): Fix \(\eps > 0\). If  \(X\) is tight then there is a compact  \(K\) with  \(\P(X \in K) > 1-\eps\). By Portmanteau:
	 \[
		 \lim\inf \P_\star(X_n \in K^\delta) \geq \P(X\in X^\delta)
	.\] 
	which is larger than \(1-\eps\) for every  \(\delta > 0\). 

	Conversely, suppose that \(X_n\) is asymptotically tight. Then there exists a compact \(K\) with  \(\lim\inf P_\star(X_n\in K^\delta) \geq 1-\eps\). By Portmanteau,
	\[
		1-\eps \leq \lim\inf \P_\star(X_n \in K^\delta) \leq \lim\sup \P^\star(X_n\in \overline{K^\delta}) \leq  \P\left(X\in \overline{K^\delta}\right)
	.\] Let \(\delta\to 0\) by monotone convergence to complete the result.
	\footnote{This proof relies on compact sets being closed in metric spaces. The proof of this is as follows: 
	Let  \(A\) be compact in a metric space. We wish to show that \(A\) is closed. Take a point \(x \in X\setminus A\). 
	To show that \(A\) is closed, we want to show that there is an open neighborhood of  \(x\) that is not in  \(A\) (this will show that \(A\) contains all of its limit points). For every  \(a \in A\), let  \(U_a = B\left(a, \frac{d(a,x)}{2} \right)\) and \(V_a = B\left(x,\frac{d(a,x)}{2} \right)\). By triangle inequality, \(U_a\) and  \(V_a\) are disjoint.
	The union of all the sets \(U_a\) for all points  \(a \in A\) is an open cover of A. By compactness of  \(A\), we can get a finite subcover \(U_{a_1},\dots,U_{a_n}\). But then \(V_{a_1}\cap \dots\cap V_{a_n}\) is an open neighborhood of \(x\)  that is disjoint from  \(A\). So \(A\) is closed. Actually this argument holds in general Hausdorff spaces.}
\end{proof}
The converse is not generally true. Let \(X_n = -1\) if  \(n\) is odd and \(X_n = 1\) if  \(n\) is even. This sequence is asymptotically measurable and asymptotically tight but clearly does not converge.  However, it does converge among a subsequence. This is the idea behind the partial converse to this theorem provided by Pohorov's Theorem.
\begin{theorem}[Pohorev's Theorem, Theorem 1.3.9 VdV\& W]
	\label{thm:pohorev}
	Let \(X_n\) be an asymptotically tight and asymptotically measurable sequence. Then there is a subsequence  \(X_{n_j}\) that converges weakly to a tight Borel law.
\end{theorem}
Now a review problem
\begin{example}[Problem 7; Ch 1.3 VdV\& W]
	\label{ex:p7}
	Let \(X_n\) be a sequence of random elements in  \(\mathbb{D}\) and \(g:\mathbb{D}\to\mathbb{E}\) a continuous function. Want to show that:
	 \begin{enumerate}
		 \item If \(X_n\) is asymptotically tight then  \(g(X_n)\) is asymptotically tight.
		 \item If \(X_n\) is asymptotically measurable then  \(g(X_n)\) is asymptotically measurable. 
	\end{enumerate}
\end{example}
\begin{proof}
	1) Suppose that \(X_n\) is asymptotically tight. Fix  \(\eps > 0\). We know that there exists a compact set  \(K\) such that, \(\forall \delta_1 > 0\)
	 \[
		 \lim\inf \P_\star\left(X_n \in K^{\delta_1}\right) \geq  1-\eps
	.\]
	The event \(\left\{X_n \in K^{\delta_1}\right\}\) is a subset of the event that \(\left\{g(X_n) \in g(K^{\delta_1})\right\}\) so
	\[
		\lim\inf \P_\star\left(g(X_n) \in g(K^{\delta_1})\right) \geq \lim\inf \P_\star\left(X_n \in K^\delta_1\right) \geq 1-\eps 
	.\] 
	To finish recall that \(g(K)\) is a compact set and choose \(\delta_1\) such that  \(g(K^{\delta_1})\subseteq g(K)^\delta\) (always possible to do so by continuity of  \(g\)).

	2) Suppose that \(X_n\) is asymptotically measurable. This means that, for any  \(f \in C_b(\mathbb{D})\):
	 \[
		 \E^\star\left[f(X_n)\right] - \E_\star\left[f(X_n)\right]\to0
	.\]
	Let \(\tilde f \in C_B(\mathbb{E})\). For any continuous \(g:\mathbb{D}\to \mathbb{E}\), \(f\circ g\) is a continuous and bounded function from  \(\mathbb{D}\to\SR\). This completes the proof.
\end{proof}

