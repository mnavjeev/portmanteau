\subsection{Symmetrization}%
\label{subsec:symmetrization}

Symmetrization is a technique that will allow us to get/show(?) a subgaussian process. This follows the discussion in Chapter 2.2.1 and 2.3 in VanDerVaart and Wellner.

What sort of variables are subgaussian? A classic example below.

\begin{definition}[Rademachar Random Variable]
	\label{def:rademachar}
	Random variable \(\eps_i:\Omega_i\to\SR\) is a Rademachar random variable if
	\(\P(\eps_i=1)=\P(\eps_i=-1)=1/2.\) 
\end{definition}

The following lemma shows that a particular process consisting of Rademachar random variables is subgaussian.

\begin{lemma}[Hoeffding's Inequality]
	\label{lemma:hoeffding}
	Let \(a_1,\dots,a_n\) be constants and \(\eps_1,\dots,\eps_n\) be independent Rademachar random variables. Then 
	\[
		\P\bigg(\big|\sum_{i=1}^n a_i\eps_i\big| > x\bigg) \leq 2e^{\frac{1}{2}\frac{x^2}{\|a\|^2}  }
	.\] 
	where \(\|a\|\) denotes the Euclidean norm of \(a\).
\end{lemma}
\begin{proof}
	(From VdV\&W, Lemma 2.2.7) For any \(\lambda\) and any Rademachar random variable \(\eps\) one has that  \(\E e^{\lambda\eps} = \left(e^\lambda+e^{-\lambda}\right)/2\). By power series expansion:
	\begin{align*}
		e^\lambda &= 1 + \lambda + \frac{\lambda^2}{2!} + \frac{\lambda^3}{3!} +\dots \\ 
		e^{-\lambda} &= 1 - \lambda + \frac{\lambda^2}{2!} - \frac{\lambda^3}{3!}+\dots \\
		\implies \left(e^{\lambda} + e^{-\lambda}\right)/2 &= 1 + \frac{\lambda^2}{2!} + \frac{\lambda^4}{4!} + \frac{\lambda^6}{6!} \dots \\
														 &\leq 1 + \frac{\lambda^2}{2} + \frac{\lambda^4}{2^2\cdot2!} + \frac{\lambda^6}{2^3\cdot3!}   \\
														 &= e^{\lambda^2/2}
	\end{align*}
	where in the last inequality we use that \(2^k\cdot k! \leq (2k)!\) so that in total we have that \(\E e^{\lambda\eps} = \left(e^\lambda + e^{-\lambda}\right)/2 \leq e^{\lambda^2/2}\). Take \(\lambda = x/\|a\|^2\) and apply Markov's inequality to get the result.
\end{proof}

\begin{example*}
	For any functions \(f,g\) we have that
	\[
		\P\bigg(\bigg|\sum_{i=1}^n \frac{\eps_i}{\sqrt{n}}\left(f(x_i)-g(x_i)\right)\bigg| > x \mid \{X_i\}  \bigg) \leq 2e^{-\frac{1}{2}\frac{x^2}{d_n^2(f,g)}  }
	.\]
	where \(d_n^2(f,g):=\frac{1}{N}\sum_{i=1}^n \left(f(x_i)-g(x_i)\right)^2 \) is the square of the prediction norm.
\end{example*}

We would like to use the maximal inequality in Theorem~\ref{thm:vdv2.2.4} to control \(\E[\sup_{f,g}|\mathbb{G}_n(f)-\mathbb{G}_n(g)|]\), but the problem is that \(\mathbb{G}_n\) is not (in general), subgaussian. However, from Lemma~\ref{lemma:hoeffding} we know that, at least conditional on our data, \(\mathbb{G}^\circ_n :=\frac{1}{\sqrt{n}}\sum \eps_i(f(x_i)-g(x_i)) \) is. Strategy will be to relate the two processes, \(\mathbb{G}_n\) and \(\mathbb{G}^\circ_n\).

Before starting, it is useful to formally define the probability space that we are working with. Let \(\eps_1,\dots,\eps_n\) be i.i.d Rademachar random variables that are generated independent of \((X_1,\dots,X_n)\), our observed data. Define the symmetrized process:
\[
	\P_n^\circ f = \frac{1}{n}\sum_{i=1}^n \eps_if(X_i)
.\] 
Because \(\P_n^\circ\) is subgaussian, conditional on  \(X_1,\dots,X_n\), it can be easier to study. We want to bound supremum of the process \(\P_n-P\) by that of the symmetrized process. To formalize these bounds, we have to be careful about the non-measurability of supremum like \(\|\P_n - P\|_\calF\).\footnote{Even if \(\calF\) is a class of measurable functions, the supremum may not be measurable.} 

In the following discussion, outer expectations of functions of \(X_1,\dots,X_n\) are assumed to be taken with respect to the coordinate projection of the infinite product space \((\calX^\SN,\calA^\SN,P^\SN)\) onto its first \(n\) coordinates,  \((\calX^n,\calA^n, P^n)\).\footnote{That is the outer expectation is taken relative to \(P^n\)  where  \(P^n\) is defined from the projection of the infinite product space onto its first  \(n\) coordinates}. When auxiliary variables, independent of the \(X\)'s are involved, as in the next lemma, we can use a similar convention. The underlying probability space is assumed to be of the form \((\calX^n,\calA^n,P^n)\times(\calZ,\calC,Q)\). Independence is understood in terms of a product probability space.\footnote{Two sub-sigma algebras, \(\calA_1, \calA_2 \subset\calA\) are considered independent if \(\P(A_1A_2) = \P(A_1)\P(A_2)\) for any  \(A_1 \in \calA_1,A_2\in\calA_2\). The sigma algebra generated by a
random map \(X:\left(\Omega,\calA,\P\right)\to(\calX,\calB)\) is the smallest sigma algebra on \(\Omega\) that makes  \(X\) measurable, 
\[
	\sigma(X) := \{X^{-1}(B):B\in\calB\} 
.\] 
Two random variables, \(X,Y\), defined on the same probability space are independent if their generated sigma algebras, \(\sigma(X),\sigma(Y)\), are independent. In the context of having independent draws \(X_1,\dots,X_n\) we can think of this as the projection mappings \(\pi_i(\calX^n)\) being independent. 
}. To manage all this, we take advantage of a modified Fubini's theorem for outer expectations, stated here without proof. 
\begin{lemma}[Fubini's Theorem, Lemma 1.2.6 VdV\&W]
	\label{lemma:vdv1.2.6}
	Let \(T\) be defined on a product probability space. Then
	 \[
		 \E_\star T\leq \E_{1\star}\E_{2\star}T\leq \E_1^\star\E_2^\star T \leq \E^\star T
	.\] 
\end{lemma}
\begin{proof}
	For the last inequality, we can assume that \(\E^\star T < \infty\) so that  \(\E^\star T = \E T^\star\). Since \(T^\star\) is jointly measurable with respect to the product  \(\sigma\)-field, the map  \(\omega_2\mapsto T^\star(\omega_1,\omega_2)\) is a measurable majorant of  \(\omega_2\mapsto T(\omega_1,\omega_2)\) for \(P_1\) almost all  \(\omega_1\). Hence \(\int T^\star(\omega_1,\omega_2)dP_2(\omega_2)\geq \left(\E_2^\star T\right)(\omega_1)\) for \(P_1\) almost all  \(\omega_1\). Further, by Fubini's theorem for standard integrals, this is a measurable function of \(\omega_1\). Thus the integral of this with respect to \(P_1\) is an upper bound for \(\E_1^\star\E_2^\star T\). Since  \(T^\star\) is jointly measurable, by another application Fubini's theorem for standard integrals:
	 \[
		 \E^\star T = \E T^\star = \int\left(\int T^\star(\omega_1,\omega_2)dP_2(\omega_2)\right)dP_1(\omega_1)\geq \E_1^\star\E_2^\star T 
	.\]
	The inequalities for inner expectations hold by considering \(-T\).
\end{proof}


\begin{lemma}[Symmetrization]
	\label{lemma:vdv2.3.1}
	For every non-decreasing, convex, \(\Phi:\SR\to\SR\) and class of measurable functions \(\calF\):
	\[
	    \E^\star\Phi\left(\left\|\P_n-P\right\|_\calF\right)\leq \E^\star\Phi\left(2\left\|\P_n^\circ\right\|_\calF\right)
	.\]
	Where outer expectations are calculated as described above.
\end{lemma}
\begin{proof}
	Let \(Y_1,\dots,Y_n\) be independent copies of \(X_1,\dots,X_n\) (independently drawn from the same joint distribution as \(X_1,\dots,X_n\), defined formally as the coordinate projections on the last \(n\) coordinates in the product space  \((\calX^n,\calA^n,P^n)\times(\calZ,\calC,Q)\times(\calX^n,\calA^n,P^n)\)). 

	For \underline{fixed} values \(X_1,\dots,X_n\) applying Jensen's inequality to the absolute value gives:
	\[
		\left\|\P_n-P\right\|_\calF = \sup_{f\in\calF}\frac{1}{n}\left|\sum_{i=1}^n \big[f(X_i)-\E f(Y_i)\big]\right|\leq \E^\star_Y \sup_{f\in\calF} \frac{1}{n}\left|\sum_{i=1}^n \big[f(X_i)-f(Y_i)]\right| 
	.\] 
	where \(\E_Y^\star\) is the outer expectation with respect to \(Y_1,\dots,Y_n\) computed for \(P^n\). Again applying Jensen's inequality gives:
	 \[
		 \Phi\left(\left\|\P_n-P\right\|_\calF\right)\leq \E_Y\Phi\left(\bigg\|\frac{1}{n}\sum_{i=1}^n \big[f(X_i)-f(Y_i)] \bigg\|_\calF^{\star Y}\right)
	.\]
	where \(f^{\star Y}\) is the minimal measurable majorant of  \(f\) with respect to the distribution of  \(Y\). Because  \(\Phi\) is non-decreasing and continuous, the  \(\star Y\) inside  \(\Phi\) can be moved to  \(\E^\star_Y\). In total then:	 \[
		\Phi\left(\left\|\P_n-P\right\|_\calF\right)\leq \E_Y^{\star}\Phi\left(\bigg\|\frac{1}{n}\sum_{i=1}^n \big[f(X_i)-f(Y_i)] \bigg\|_\calF\right)
	.\]
	Next, take the expectation with respect to \(X_1,\dots,X_n\) of the above quantity to get:
	\[
		\E^\star\Phi\left(\|\P_n-P\|_\calF\right) \leq \E_X^\star\E_Y^\star\Phi\left(\frac{1}{n}\bigg\|\sum_{i=1}^n \big[f(X_i)-f(Y_i)\big]\bigg\| \right)
	.\]
	Adding a minus sign in front of the term \(\big[f(X_i)-f(Y_i)\big]\) has the effect of exchanging  \(X_i\) and \(Y_i\). By construction of the underlying probability space this does not change the expectation. Hence, the expression
	 \[
	    \E^\star\Phi\left(\frac{1}{n}\bigg\|\sum_{i=1}^n e_i\big[f(X_i)-f(Y_i)\big]\bigg\| \right)
	.\]
	is the same for any \(n\)-tuple  \((e_1,\dots,e_n)\in\{-1,1\}^n\). So:
	\[
		\E^\star\Phi\left(\big\|\P_n-P\big\|_\calF\right)\leq \E_\eps\E^\star_{X,Y}\Phi\left(\bigg\|\sum_{i=1}^n \eps_i\big[f(X_i)-f(Y_i)\big]\bigg\|_\calF\right)
	.\]
	where each \(\eps_i\) is an independent Rademachar random variable and  \(\eps = (\eps_1,\dots,\eps_n)\). By triangle inequality and convexity of the \(\Phi\):
	\begin{align*}
		\E_\eps\E^\star_{X,Y}&\Phi\left(\bigg\|\sum_{i=1}^n \eps_i\big[f(X_i)-f(Y_i)\big]\bigg\|_\calF\right)\\ 
		&\leq \E_\eps\E^\star_{X,Y}\Phi\left(\bigg\|\frac{1}{n}\sum_{i=1}^n \eps_if(X_i)\bigg\|_\calF + \bigg\|\frac{1}{n}\sum_{i=1}^n \eps_if(Y_i) \bigg\|_\calF\right)\\
		&\leq  \frac{1}{2}\E_\eps\E^\star_{X,Y}\Phi\left(2\bigg\|\frac{1}{n} \sum_{i=1}^n \eps_if(X_i)\bigg\|\right) + \frac{1}{2}\E_\eps\E^\star_{X,Y}\Phi\left(2\bigg\|\frac{1}{n} \sum_{i=1}^n \eps_if(Y_i)\bigg\|\right) \\
		&\leq \E^\star\Phi\left(2\left\|\P_n^\circ\right\|_\calF\right)
	\end{align*}
	where we use the face that a repeated outer expectation can be bounded above by a joint outer expectation, \(\E_\eps\E^\star_{X,Y} \leq \E^\star_{\eps,X,Y} (=\E^\star)\) using Lemma~\ref{lemma:vdv1.2.6}.
\end{proof}

\begin{corollary}[Symmetrization of Empirical Process, Andres' Notes]
    \label{corr:symmetrization}
	For real valued processes as described above:
	\[
		\E^\star\left[\sup_{f\in\calF}\bigg|\frac{1}{\sqrt{n}}\sum_{i=1}^n f(X_i)- P f(X_i)\bigg| \right] \leq 2\E^\star \left[\sup_{f\in\calF}\bigg|\frac{1}{\sqrt{n}}\sum_{i=1}^n \eps_if(X_i)\bigg|\right]
	.\] 
\end{corollary}
\begin{proof}
	Take \(\Phi(x) = |x|\). All norms on \(\SR\) are equivalent to \(|\cdot|\). Lemma~\ref{lemma:vdv2.3.1} then gives us that \(\E^\star\left\|\P_n - P\right\|_\calF \leq 2\E^\star\left\|\P_n^\circ\right\|_\calF\). Expand this out and scale by \(\sqrt{n}\) to get the result. For non real-valued processes (vector valued, function valued, etc.) we can replace \(\left|\cdot\right|\) with \(\left\|\cdot\right\|\) above.
\end{proof}
\begin{remark*}
	The proof of Corollary~\ref{corr:symmetrization} uses the fact that Lemma~\ref{lemma:vdv2.3.1} is not an asymptotic bound, it holds in every finite sample. 
\end{remark*}

We know have the pieces to show a class of functions \(\calF\) is either
\begin{itemize}
	 \item Glivenko-Cantelli, i.e that  \(
			\left\|\P_n-P\right\|_\calF = o_p(1)
			.\) We will do this by placing conditions on the bracketing/covering numbers.
	\item Donsker, i.e that \(\mathbb{G}_n(\calF) \overset{L}{\to}\mathbb{G}(\calF)\) for some tight \(\mathbb{G}\). To do so, we will use covering numbers. The system of arguments needed to show this is usually as follows:
	\begin{itemize}
		\item By Theorem~\ref{thm:vdv1.5.4} weak convergence to a tight limit is equivalent to asymptotic tightness and weak convergence of the marginals. 
		\item Weak convergence of the marginals is generally provided by CLT.Theorem~\ref{thm:vdv1.5.7} shows that asymptotic tightness is equivalent to uniform \(\rho\)-equicontinuity (Definition~\ref{def:equicontinuous})
		\item Asymptotic equicontinuity holds if \(\E\left[\sup_{f\in\calF_\delta}\left|\mathbb{G}_n(f)\right|\right]\) goes to 0 as \(\delta\downarrow 0\). Theorem~\ref{thm:vdv2.2.4} gives conditions where this is possible for separable, subgaussian processes.

		\item Lemma \ref{lemma:continuous-separable} suggests separability if \(\calF\) is separable. Lemma~\ref{lemma:hoeffding} gives us that the Rademachar process is subgaussian conditional on \(X_1,\dots,X_n\). Combining with Theorem~\ref{thm:vdv2.2.4} gives

		\[
			\E_\eps\left[\sup_{f\in\calF_\delta}\bigg|\frac{1}{\sqrt n}\sum_{i=1}^n \eps_i f(X_i)\bigg| \right] \leq \int_{0}^{\text{diam}(\calF_\delta)}\sqrt{\log\calN\left(s,\calF_\delta,L_2(\P_n)\right)}\; ds
		\]
		where \(L_2(\P_n) = \frac{1}{n}\sum_{i=1}^n f(X_i)^2 \) is the \(L_2\) norm with respect to the empirical measure. Since \(X\) is random this norm will also end up random. This seems like it will make dealing with
			\[
				\E\left[\int_{0}^{\text{diam}(\calF_\delta)}\sqrt{\log\calN\left(s,\calF_\delta,L_2(\P_n)\right)}\;ds  \right]
			\] 
			painful, but we end up having good bounds for this.
		\item Lemma~\ref{lemma:vdv2.3.1}, and in particular Corollary~\ref{corr:symmetrization}, relates the empirical process to the Rademachar process. Take expectations with respect to \(X\) in the above bound to bound the and apply the symmetrization lemma to get bounds on the empirical process of interest.
	\end{itemize}
\end{itemize}
We next move to verifying the various conditions and applying them to show that some specific processes are Glivenko-Cantelli or Donsker.

\subsection{Glivenko-Cantelli}%

This subsection follows Section 2.4 in Van DerVaart and Wellner. Goal is to establish conditions for a uniform law of large numbers using bracketing and covering numbers.

\begin{theorem}[Theorem 2.4.1 VdV\&W]
	\label{thm:vdv2.4.1}
	Let \(\calF\) be a class of measurable functions such that  \[\calN_{[\hspace{0.1em}]}\left(\eps,\calF,L_1(P)\right)<\infty\] for every \(\eps>0\). Then  \(\calF\) is Glivenko-Cantelli. 
\end{theorem}
\begin{proof}
	Fix \(\eps > 0\). Choose finitely many \(\eps\)-brackets \([l_i,u_i]\) whose union contains \(\calF\) and such that \(P(u_i-l_i)<\eps\) for every \(i\). Then, for every \(f\in\calF\) there is a bracket, \(l_i \leq f \leq u_i \), such that:
	\begin{align*}
		\left(\P_n-P\right)f \leq \P_n u_i - Pf\leq \left(\P_n-P\right)u_i + P\left(u_i - f\right) \leq \left(\P_n-P\right)u_i + \eps \\ 
		\left(\P_n-P\right)f \geq \P_n l_i - Pf\geq \left(\P_n-P\right)l_i + P\left(l_i - f\right) \geq \left(\P_n-P\right)l_i - \eps 
	\end{align*}
	Consequently,
	\begin{align*}
			\sup_{f\in\calF}\left(\P_n-P\right)f &\leq \max_i \left(\P_n-P\right)u_i + \eps \\
			\inf_{f\in\calF}\left(\P_n-P\right)f &\geq \min_i \left(\P_n-P\right)l_i -\eps
	\end{align*}
	By the strong law of large numbers, both the maximums and the minimums on the right hand side of the inequalities above converge almost surely to 0. Combination these yields that \(\lim\sup\left\|\P_n-P\right\|^\star_\calF \leq \eps\) almost surely for every \(\eps > 0\). Take  \(\eps\downarrow 0\) to see that the \(\limsup\) must be  \(0\) almost surely. 
\end{proof}
\begin{remark*}
	Some comments on Theorem~\ref{thm:vdv2.4.1}:
	\begin{enumerate}
		\item Proof is really quite straightforward. Bracketing gives pointwise control so just use the upper and lower bounds.
		\item No measurability condition is needed and no requirements on the rate of growth of \(\calN_{[\hspace{0.1em}]}\left(\eps,\cdot,\cdot\right)\) as \(\eps\downarrow 0\).
	\end{enumerate}
\end{remark*}
\begin{example*}[Empirical CDF is Glivenko-Cantelli]
	Let \(X\) be a scalar random variable.\footnote{This generalizes easily for a vector valued random variable}. We want to show that 
	\[
		\sup_{t\in\SR}\bigg| \frac{1}{n}\sum_{i=1}^n \mathds{1}\{X_i\leq t\}-P(X_i\leq t)\bigg|=o_p(1)
	.\]
	Let \(\calF = \left\{f(x)=\mathds{1}\{X_i\leq t\}: t \in \SR\right\}\). Partition \(\SR\) into grids  \(-\infty = t_0 < t_1 < \dots< t_m = \infty\) such that \(\P\left(t_i \leq X \leq t_{i+1}\right)<\eps\) for each \(i\). Then the finitely many brackets \(\left[\mathds{1}\{X_i \leq t_i\},\mathds{1}\{X_i \leq t_{i+1}\}\right]\) cover \(\calF\) and are ``size'' \(\eps\) under \(P\). So, \(\calN_{[\hspace{0.1em}]}\left(\eps,\calF,L_1(P)\right)<\infty\) for every \(\eps>0\). So \(\calF\) is Glivenko-Cantelli (i.e, we have a uniform law of large numbers).
\end{example*}
The requirement on the bracketing numbers can in general be hard. Would like a result for the covering numbers as well. This will make showing that some classes are Glivenko-Cantelli easier later on. Before doing so, we need to make a couple definitions:
\begin{definition}[Envelope]
	\label{def:envelope}
	A class \(\calF\) has envelope \(F\) if \(|f(x)|\leq F(x)\) for all \(x\) and all \(f\in\calF\).
\end{definition}
\begin{definition}[Truncated Class]
	\label{def:truncated-class}
	Let \(\calF\) be a class of functions. Then the truncated class \(\calF_M\) is given 
	\[
		\calF_M = \left\{f(x)\mathds{1}\{f \leq M\}: f\in\calF\right\}	
	.\]
\end{definition}
\begin{definition}[P-Measurable Class]
	\label{def:p-measurable-class}
	A class \(\calF\) is \(P\)-measurable if \(\sup_{f\in\calF}\left|\frac{1}{n}\sum_{i=1}^n f(x_i)\eps_i\right|\) is measurable with respect to the product measure on \((\calX^m,\calA^m,P^n)\times(\calZ,\calC,Q)\), where \((\calZ,\calC,Q)\) denotes the probability space that the Rademachar random variables are defined on.
\end{definition}
\begin{definition}[\(L_p(\P_n)\)-norm]
	\label{def:lpn-norm}
	We have that \(\left\|f-g\right\|_{L_1(P)} = \E_P\left[\left|f(x)-g(x)\right|\right]\), similarly we can define 
	\[
			\left\|f-g\right\|_{L_{1(\P_n)}} = \E_{\P_n}\left[\left|f(x)-g(x)\right|\right]
	.\]
	and through this define \(\calN_{[\hspace{0.1em}]}\left(\eps,\calF,L_1(\P_n)\right)\).
\end{definition}

\begin{theorem}[Theorem 2.4.3, VdV\&W]
	\label{thm:vdv2.4.3}
	Let \(\calF\) be a  \(P\)-measurable class of measurable functions with envelope \(F\) such that \(\P^\star F<\infty\). If  \(\log\calN\left(\eps,\calF_M,L_1\left(\P_n\right)\right)= o_{P^\star}(n)\) for every \(\eps\) and  \(M > 0\), then  \(\left\|\P_n-P\right\|^\star_\calF\to 0\) almost surely and in mean. 		
\end{theorem}



