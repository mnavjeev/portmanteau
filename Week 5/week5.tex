\section{Empirical Processes}

These notes follow Section 2 in VdV\&W. So far, we have discussed theory for \(X_n \overset{L}{\to} X\) where both \(X_n\) and  \(X\) are random elements in  \(\ell^\infty(T)\). The classic example that we have kept in mind is convergence of the empirical CDF process, \(X_n(t) = \frac{1}{\sqrt{n}}\sum_{i=1}^n \left(\mathds{1}\{X_i \leq t\} - \P(X\leq t)\right)\). In this next section we will build on the theory developed to show the convergence of some empirical processes on \(\ell^\infty\).

\begin{definition}[Empirical Measure]
	\label{def:empirical-measure}
	For a random sample \(\{X_i\}_{i=1}^n\), the empirical measure \(\P_n\) is the measure constructed from the sample (putting mass \(1/n\) at each  \(X_i\)). That is, for any set \(C\):
	 \[
		 \P_n(C) := \frac{1}{n}\sum_{i=1}^n \mathds{1}\{X_i \in C\}  
	.\]
	We can also write this in terms of the degenerate measures on each \(X_i\):
	 \[
		 \P_n := \frac{1}{n}\sum_{i=1}^n \delta_{X_i} 
	.\] 
\end{definition}

\begin{definition}[Empirical Process]
	\label{def:empirical-process}
	For a random sample \(\{X_i\}_{i=1}^n\) drawn from common distribution \(P\), the empirical process \(\mathbb{G}_n\) is the scaled and demeaned measure on \(X\) given by:
	\[\mathbb{G}_n(C) := \frac{1}{\sqrt{n}}\sum_{i=1}^n \left(\mathds{1}\{X_i \in C\} - P(X_i \in C)\right).\]
	This is often related to the empirical measure in Definition~\ref{def:empirical-measure} by
	\[
		\mathbb{G}_n = \sqrt{n}\left(\P_n - P\right)
	.\]
	Or written in terms of the degenerate measures on each \(X_i\):
	 \[
		 \mathbb{G}_n = \frac{1}{\sqrt{n}}\sum_{i=1}^n \left(\delta_{X_i}-P\right) 
	.\] 
\end{definition}

\begin{remark}[Notation]
    \label{rem:ep-notation}
	We will make the following notations to save space later on. For a measure \(\mathbb{Q}\) on a space let \(\mathbb{Q}f = \E_{\mathbb{Q}}[f(X)]\). In the above \(\P_nf = \E_{n}[f(X)] = \frac{1}{n}\sum_{i=1}^n f(X_i) \) and \(\mathbb{G}_n f = \frac{1}{\sqrt{n}}\sum_{i=1}^n \left(f(X_i) - Pf\right)\).

	With this notation:
	\begin{align*}
		\P_nf\overset{\text{a.s}}{\longrightarrow}Pf&\hbox{  }\text{ is just saying }\hbox{ }\frac{1}{n}\sum_{i=1}^n f(X_i)\overset{\text{a.s}}{\longrightarrow}\E\left[f(X)\right]\\ 
		\mathbb{G}_nf\overset{L}{\longrightarrow}N(0,\sigma^2)&\hbox{  }\text{ is just saying }\hbox{  }\frac{1}{\sqrt{n}}\sum_{i=1}^n \left(f(X_i)-\E\left[f(X)\right]\right)\overset{L}{\longrightarrow}N(0,\sigma^2) 
	\end{align*}
	By LLN and CLT we have that for any function \(f\),  \(\P_nf \to_{a.s}Pf\) and \(\mathbb{G}_nf \overset{L}{\to}N\left(0,P\left(f-Pf\right)^2\right)\)
\end{remark}

\begin{example}[Classes of Functions]
	\label{ex:calF}
	LLN and CLT establish the behavior of the empirical measure \(\P_nf\) and the empirical process \(\mathbb{G}_nf\) for a fixed function \(f\) (which could even be vector valued). However, we often want to study the behavior of the empirical measure of empirical process over a class of functions  \(\calF\). In this case we can think of  \(\mathbb{G}_n(\calF)\) or \(\P_n(\calF)\) as random maps onto \(\ell^\infty(\calF)\). The marginal, \(\mathbb{G}_nf\) or \(\P_nf\), is then the behavior of the empirical measure/process for a single function \(f\in\calF\).

	Mapping this back to the empirical CDF example of before let \(\calF = \left\{f_t:\SR\to \SR\mid f_t(x) = \mathds{1}\{x\leq t\}, t \in T\right\}\). Before, we considered convergence of the whole CDF through the map \(X_n:\Omega_n \to \ell^\infty(T)\) with the marginals \(X_n(t) = \frac{1}{n}\sum_{i=1}^n \mathds{1}\{X_i \leq t\}  \). With these new definitions/notations, we equivalently consider convergence of the entire CDF through the map \(\P_n(\calF):\Omega_n \to \ell^\infty(\calF)\) with marginals \(\P_nf_t = \frac{1}{n}\sum_{i=1}^n \mathds{1}\{X_i \leq t\} \).

	This sort of notation/generality is useful as we can consider the behavior of the empirical measure or empirical process over a larger class of functions. For example, if we wanted to study an entire semiparametric model we may consider the behavior of \(\mathbb{G}_n(\calF)\) where 
	\[
		\calF = \left\{f(x;\theta)\text{ for some }\theta\in\Theta\right\}
	.\]
	Or, if we wanted to consider convergence after imposing some shape restriction, we may take
	\[
		\calF = \left\{f:X\to\SR\mid f\text{ is monotonic}\right\}
	.\] 
\end{example}

\begin{remark}[Notation]
	Sometimes we use \(\wcov\) to denote weak convergence/convergence in law instead of  \(\overset{L}{\to}\).	
\end{remark}

\begin{remark}[Definition of \(\ell^\infty\) Space]
    \label{rem:def-recall}
	It is useful to review the \(\ell^\infty(T)\) space for an arbitrary index space  \(T\). Define:
	\begin{equation}
		\label{eq:ell-infty}
		\ell^\infty(T) = \left\{f:T\to\SR: \sup_{t\in T}\left|f(t)\right| <\infty\right\}
	\end{equation}
	and equip this space with the sup-norm, \(\left\|f\right\|_T = \sup_{t\in T}\left|f(t)\right|\). Note that, for any \(\calF\),  \(\mathbb{G}_n(\calF)\) can be viewed as a random map into \(\ell^\infty(\calF)\) for each \(n\). Boundedness comes from the finiteness of the sample. We will sometimes make the notation \(\left\|\mathbb{Q}\right\|_\calF = \sup_{f\in\calF}\left|\mathbb{Q}f\right|\) for a given measure \(\mathbb{Q}\).
\end{remark}

Now make some important definitions and then talk about how they relate to what we want to show. 

\begin{definition}[Glivenko-Cantelli Class]
	\label{def:gc-class}
	A class of functions, \(\calF\), for which
	\begin{equation}
		\label{eq:gc-def}
		\left\|\P_n-P\right\|_\calF \to_p 0
	\end{equation}
	is called a Glivenko-Cantelli class, or a \(P\)-Glivenko-Cantelli class to emphasize the dependence on the underlying measure \(P\) from which the sample is drawn.
\end{definition}

\begin{definition}[Donsker Class]
	\label{def:donsker-class}
	A class of functions, \(\calF\), for which
	 \begin{equation}
		\label{eq:donsker-def}
		\mathbb{G}_n(\calF) \overset{L}{\longrightarrow} \mathbb{G}(\calF)
	\end{equation}
	where \(\mathbb{G}\) is a tight, Borel measurable element in \(\ell^\infty(\calF)\), is called a Donsker class, or  \(P\)-Donsker class to emphasize the dependence on the underlying measure  \(P\) from which the sample is drawn.
\end{definition}

A Donsker class is trivially Glivenko-Cantelli. 

\begin{example}[Some Donsker Classes]
	Some examples of function classes:
	\begin{enumerate}
		\item If \(\calF\) consists of a single function with finite variance then \(\calF\) is Donsker by the Central Limit Theorem. That is  \(\mathbb{G}_n \overset{L}{\to}\mathbb{G}\) where \(\mathbb{G}\) is a tight element on \(\ell^\infty(\calF) = \ell^\infty(\{f\} )\) 
		\item The class of functions \(\calF = \left\{f(x) = x'\beta: \beta \in \calB\right\} \) is Donsker if \(\calB\) is bounded. 
		\item The class of monotonic densities on \([0,1]\) is Donsker. 
		\item The class of square integrable functions is not Donsker (too large).
	\end{enumerate}	
\end{example}

How do we know if \(\mathbb{G}_N \wcov \mathbb{G}\) where \(\mathbb{G}\) is a tight, Borel measurable element on \(\ell^\infty(\calF)\)? By Theorem~\ref{thm:vdv1.5.4} we know that \(X_n\) weakly converges if and only if \(X_n\) is asymptotically tight and the marginals  \(\left(X_n(t_1),\dots,X_n(t_k)\right)\) converge weakly to a limit for every finite subset. Moreover, by Lemma~\ref{lemma:vdv1.5.1} asymptotic measurability of the process is equivalent to asymptotic measurability of the marginals. By the Central Limit Theorem, we typically have weak convergence and asymptotic measurability of the marginals, what remains is to show asymptotic tightness. 

Theorem~\ref{thm:vdv1.5.7} characterizes asymptotic tightness in terms of \(\rho\)-equicontinuity. Much of the work in showing tightness will be to find some semimetric \(\rho\) on \(\calF\) such that for any \(\eps, \eta > 0\) there is a  \(\delta > 0\) such that 
\begin{equation}
	\label{eq:rho-ec}
	\lim\sup_{n\to\infty}\P^\star\left(\sup_{\rho(f,g)<\delta}\left|\mathbb{G}_n(f)-\mathbb{G}_n(g)\right| > \eps\right) <\eta.
\end{equation}

A typical approach will be to let \(\calF_{{}\delta} = \left\{f,g \in \calF, \rho(f,g)<\delta\right\}\). If we can show that, for some \(M(\delta)\) that goes to 0 as \(\delta\downarrow 0\)
\begin{align*}
	\E\left[\left\|\mathbb{G}_n\right\|_{\calF_{{}\delta}}\right] 
	&= \E\left[\sup_{\rho(f,g)<\delta}\left|\mathbb{G}_n(f)-\mathbb{G}_n(g)\right|\right] \\
	&= \E\left[\sup_{\rho(f,g)<\delta}\left|\frac{1}{\sqrt{n}} \sum_{i=1}^n\left\{ f(X_i) - \E[f(X_i)] - g(X_i) + \E[g(X_i)]\right\}\right|\right] \\
	&\leq M(\delta)
\end{align*}
Then, we would get the result in \eqref{eq:rho-ec} by Markov's inequality. This type of result, that \(\E\left[\left\|\mathbb{G}_n\right\|_{\calF_{{}\delta}}\right] \leq M(\delta)\) is called a maximal inequality and is immensely useful.

Obtaining such a maximal inequality/establishing asymptotic tightness is dependent on the space not being ``too large" (loosely speaking). In the example above, the class \(\calF = \{f(x) = x'\beta\mid \beta\in\calB\} \) is Donsker so long as \(\calB\) is bounded. To illustrate, see in the single dimensional case that
\[
	\sup_{b\in \calB}\left|\frac{1}{\sqrt{n}}\sum_{i=1}^n x_ib - \E[xb] \right| = \sup_{b\in\calB} \left|\frac{1}{\sqrt{n}}\sum_{i=1}^n x_i - \E[x] \right||b|
.\]
If we don't impose \(|b|\leq M\) then this will blow up to \(+\infty\) with probability 1, whereas if we do we have that this is  \(O_p(1)\). For more involved function classes, we want a way of measuring whether  \(\calF\) is large or not. This motivates the definitions of bracketing and covering numbers below.

\begin{definition}[Covering Number]
	\label{def:covering}
	The covering number, \(\calN\left(\eps,\calF,\|\cdot\|\right)\) of a class of functions \(\calF\) is the smallest number of balls of radius  \(\eps\) under  \(\|\cdot\|\) needed to cover the set \(\calF\).
\end{definition}

\begin{definition}[Bracketing Number]
	\label{def:bracketing}
	Given two functions, \(\ell\) and \(u\), the bracket \([\ell,u]\) is the set of all functions  \(f\) with  \(\ell(x)\leq f\leq u(x)\) for all \(x\). An  \(\eps\)-bracket is a bracket  \([\ell,u]\) with  \(\|u-\ell\|< \eps\). The bracketing number \(\calN_{[]}\left(\eps,\calF,\|\cdot\|\right)\) is the minimum number of \(\eps\)-brackets needed to cover  \(\calF\).
\end{definition}

\begin{example}[Covering Number]
	\label{ex:covering}
	Let \(A = [0,1]\) and  \(\|\cdot\|\) be the standard Euclidean norm\footnote{If we want to view this as a function class we can equivalently say \(A\) is the set of constant functions taking values in the interval  \([0,1]\) and consider any \(L_p\) norm on this class}.
	\begin{enumerate}
		\item If \(\eps \geq 1/2\), then a ball centered at  \(1/2\) covers the entire interval so  \(\calN\left(\eps,A,\left|\cdot\right|\right) = 1\).
		\item If \(\eps < 1/2\), then we need \(\lceil \frac{1}{2\eps}\rceil\) balls to cover \(A\).
	\end{enumerate}
	Note that (i) in this example the covering number coincides with the bracketing number (ii) in general the balls needed to cover \(\calF\) need not be centered at points in \(\calF\) (iii) (in general) as \(\eps\downarrow 0\) we have that  \(\calN(\eps,\calF,\|\cdot\|)\uparrow\infty\). 
\end{example}

\begin{example}[Bracketing Number]
	\label{ex:brackering}
	Suppose \(x\) takes values in  \([0,1]\) and let \(\calF = \left\{f(x)=x\beta,\text{ for }\beta \in [0,1]\right\}\). Then, if \(\beta_i < \beta_{i+1}\),  \([x\beta_i, x\beta_{i+1}]\) forms a bracket containing all functions  \(f(x)=x\beta\) with  \(\beta_i \leq \beta\leq \beta_{i+1}\). Further note that \[
		\left\|x\beta_i - x\beta_{i+1}\right\| = \sup_{x\in[0,1]}|x||\beta_i-\beta_{i+1}| = |\beta_i -\beta_{i+1}|
	.\] 
	For any \(\eps>0\) break up  \([0,1]\) into  \([0,\eps,2\eps,\dots]\) and take \(\beta_i = (i-1)\eps\) to get brackets  \([x\beta_i,x\beta_{i+1}]\) of size \(\eps\). We need  \(\lceil 1/\eps\rceil\) of these brackets to cover  \(\calF\) so that  \(\calN_{[]}\left(\eps,\calF,\|\cdot\|_\infty\right)\leq \lceil 1/\eps\rceil < 2/\eps\).
\end{example}

\begin{remark}[Bracketing vs. Covering Numbers]
    \label{rem:bracketing-covering}
	In general we have that \(\calN(\eps,\calF,\|\cdot\|)\leq \calN_{[]}(2\eps,\calF,\|\cdot\|)\), but no opposite relationship. This shows that bracketing numbers are in general stronger than covering numbers and give you better control over the class of functions. 

	We will see conditions for Glivenko-Cantelli and Donsker properties under both, but in general proving Glivenk-Cantelli involves using bracketing numbers wheras proving Donsker involves using covering numbers.
\end{remark}


In general, finding the covering/bracketing number will be difficult but we will learn some tips. Verifying that a set is Donsker will often come down to showing that the covering/bracketing number does not go to infinity ``too fast." 

\subsection{Maximal Inequality}%
\label{subsec:maximal}

For an arbitrary set of functions, \(\calF\), want to develop an inequality that looks something like:
\[
	\E\left[\sup_{f\in\calF}\left|\frac{1}{\sqrt{n}}\sum_{i=1}^n \left(f(x_i)-\E[f(x)]\right) \right|\right]\leq \text{size}\left(\calF\right)
.\]
Or, rewriting in the notation of above:
\[
	\E\left[\sup_{f\in\calF}\left|\mathbb{G}_nf\right|\right]\leq \text{size}\left(\calF\right)
.\] 
This sort of inequality is useful as it can be used to show the uniform law of large numbers:
\[
	\E\left[\sup_{f\in\calF}\left|\left(\P_n - P\right)f\right|\right] \leq \frac{1}{\sqrt{n}}\text{size}\left(\calF\right)\hbox{  }\text{ + }\hbox{  }\text{Markov's Inequality}
.\] 
Or show asymptotic tightness through stochastic equicontinuity:
\[
	\E\left[\sup_{\rho(f,g)<\delta}\left|\mathbb{G}_n\left(f-g\right)\right|\right] \leq \text{size}\left(\calF_\delta\right)\hbox{  }\text{ + }\hbox{  }\text{Theorem~\ref{thm:vdv1.5.7}}
.\]
However, often we may need to change the exact application of these maximal inequalities. We will work out where these come from as we go along. The inequality will be presented for general stochastic processes (for our purposes, a stochastic process is a random map into \(\ell^\infty(T)\)). To build the maximal inequality, we will need to define a new norm which generalizes the \(L_p\) norms. We do so quickly below.

\subsubsection{Orlicz Norm}

\begin{definition}[Orlicz Norm]
	\label{def:orlicz}
	Let \(\psi\) be a nondecreasing, convex function with  \(\psi(0)=0\) and  \(X\) a random variable. Then, the Orlicz norm  \(\left\|X\right\|_\psi\) is defined as 
	\begin{equation}
		\label{eq:orlicz}
		\left\|X\right\|_\psi = \inf\left\{C > 0 : \E\psi\left(\frac{|X|}{C} \right)\leq 1\right\}
	\end{equation}
	Where here the infimum over the empty set is taken to be \(+\infty\).
\end{definition}
\begin{remark}[Orlicz norms generalize \(L_p\)]
    \label{rem:ell-orlicz}
	Note that for any \(p\geq 1\) the function \(f(x) = x^p\) is convex and non-decreasing. With this in mind we can view the Orlicz norms as a generalization of the \(L_p\) norms to general convex and nondecreasing functions functions.
\end{remark}

\begin{remark}[Orlicz p-norms]
    \label{rem:orlicz-p-norm}
	Of particular interest will be the Orlicz norms generated by the functions
	\[
		\psi_p = e^{x^p}-1
	.\] 
	for \(p\geq 1\). The Orlicz norm in this case is often denoted \(\left\|\cdot\right\|_{\psi_p}\).  These norms give more weight to the tails of \(X\) than the standard  \(L_p\) norms. It is not the case that these norms are uniformly larger than all \(L_p\) norms, however, we do have the inequalities
	\begin{align*}
		\left\|X\right\|_{\psi_p} &\leq \left\|X\right\|_{\psi_q}\left(\log 2\right)^{p/q}\\ 
		\left\|X\right\|_p &\leq p!\left\|X\right\|_{\psi_1}
	\end{align*}
\end{remark}

\begin{remark}[Orlicz Norms and Markov's Inequality]
    \label{rem:orlicz}
	Any Orlicz norm can be used to bound tail probabilites. Using Markov's inequality:
	\[
		\P\left(\left|X\right|>x\right)\leq \P\left(\psi\left(|X|/\left\|X\right\|_\psi\right)\geq \psi\left(x/\left\|X\right\|_\psi\right)\right) \leq \frac{1}{\psi\left(x/\|X\|_\psi\right)} 
	.\] 
	For \(\psi_p(x) = e^{x^p}-1\) this leads to tail estimates like  \(\exp\left(-Cx^p\right)\) for any random variable with a finite \(\psi_p\)-norm. Conversely, an exponential tail bound of this type shows that  \(\left\|X\right\|_{\psi_p}\) is finite.
\end{remark}

\begin{lemma}[Lemma 2.2.1 VdV\&W]
	\label{lemma:vdv2.2.1}
	Let \(X\) be a random variable  with  \(\P\left(|X|>x\right)\leq Ke^{-Dx^p}\) for every \(x\) and some (fixed) constants  \(K\) and  \(D\) and for some  \(p \geq 1\). Then, the Orlicz norm of \(X\) satisfies  \[\|X\|_{\psi_p}\leq \left((1+K)/D\right)^{1/p}\]
	In particular, this will mean that for \(C = \left((1+K)/D\right)^{1/p}\)
	\[
	    \E\left[\psi\left(\frac{|X|}{C}\right)\right]\leq 1
	.\] 
\end{lemma}
\begin{proof}
	By Fundamental Theorem of Calculus and Fubini's Theorem, for any constant \(B\):
	\begin{align*}
		\E\left[e^{B|X|^p}-1\right]= \E\int_0^{|X|^p}Be^{Bs}\,ds = \int_0^\infty \P\left(|X> s^{1/p}\right)Be^{Bs}\,ds
	\end{align*}
	Now use the inequality on the tails of \(|X|\), plug in \(B = C^{-p}= D/(1+K)\), and see that the final equality is bounded by 1.
\end{proof}

Using the fact that \(\max|X_i|^P \leq \sum |X_i|^p\) we obtain for the \(L_p\) norms, the result that 
\[
	\left\|\max_{1\leq i\leq m}X_j\right\|_p = \left(\E\max_{1\leq i\leq m}\left|X_i\right|^p\right)^{1/p} \leq m^{1/p}\max_{1\leq i\leq m}\left\|X_i\right\|_p
.\]
We can generalize this for the Orilcz norm.
\begin{lemma}[Lemma 2.2.2 VdV\&W]
	\label{lemma:vdv2.2.2}
	Let \(\psi\) be a convex, nondecreasing, nonzero function with  \(\psi(0)=0\) and  \(\lim\sup_{x,y\to\infty}\psi(x)\psi(y)/\psi(cxy)<\infty\) for some constant \(c\). Then, for any random variables  \(X_1,\dots,X_m\),
	\begin{equation}
		\label{eq:vdv2.2.2}
		\left\|\max_{1\leq i\leq m}X_i\right\|_\psi \leq K\psi^{-1}(m)\max_{1\leq i\leq m}\left\|X_i\right\|_\psi
	\end{equation}
	For a constant \(K\) depending only on  \(\psi\).
\end{lemma}

\begin{proof}
	Without loss of generality, assume that  \(\psi(x)\psi(y)\leq \psi(cxy)\) for all \(x,y\geq 1\).\footnote{If this is not the case there are constants \(\sigma \leq 1\) and \(\tau > 0\) such that  \(\phi(x)=\sigma\psi(\tau x)\) satisfies these conditions. Apply the inequality to  \(\phi\) and note that 
	 \[
		 \|X\|_\psi \leq  \|X\|_\phi/(\sigma\tau)\leq \|X\|_\psi/\sigma
	.\] }
	In this case, \(\psi(x/y)\leq \psi(cx)/\psi(y)\) for all \(x\geq y\geq 1\). Thus, for \(y\geq 1\) and any \(C\),
\end{proof}

